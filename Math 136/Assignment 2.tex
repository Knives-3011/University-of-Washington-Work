\documentclass[12pt]{article}
\usepackage{bigints}
\usepackage{graphicx}			% Use this package to include images
\usepackage{amsmath}	
\usepackage{amssymb}
\usepackage{amsfonts}
\usepackage{polynom}
\usepackage{listings}
% A library of many standard math expressions
\graphicspath{ {./Images/} }
\usepackage[margin=1in]{geometry}% Sets 1in margins. 
\newcommand{\qed}[0]{$\blacksquare$}
\usepackage{fancyhdr}			% Creates headers and footers
\usepackage{enumerate}          %These two package give custom labels to a list
\usepackage[shortlabels]{enumitem}


% Creates the header and footer. You can adjust the look and feel of these here.
\pagestyle{fancy}
\fancyhead[l]{Aditya Gupta}
\fancyhead[c]{Math 136 Homework \#2}
\fancyhead[r]{\today}
\fancyfoot[c]{\thepage}
\renewcommand{\headrulewidth}{0.2pt} %Creates a horizontal line underneath the header
\setlength{\headheight}{15pt} %Sets enough space for the header
\begin{document}
\begin{enumerate}
\item Let the rotation matrix around the vector \(\vec{v} = (1, 2, 3)^T\) by an angle \(\alpha\) be given by:

\[
R = P \cdot R_z(\alpha) \cdot P^T
\]

where \(R_z(\alpha)\) is the standard rotation matrix about the \(z\)-axis:

\[
R_z(\alpha) =
\begin{bmatrix}
\cos\alpha & -\sin\alpha & 0 \\
\sin\alpha & \cos\alpha & 0 \\
0 & 0 & 1
\end{bmatrix}
\]

and \(P\) is the change of basis matrix formed by the orthonormal set \(\{\vec{v}_1, \vec{v}_2, \vec{u}\}\), where:
- \(\vec{u}\) is the unit vector in the direction of \(\vec{v}\),
- \(\vec{v}_1\) is a unit vector orthogonal to \(\vec{u}\),
- \(\vec{v}_2 = \vec{u} \times \vec{v}_1\)

Thus,

\[
P =
\begin{bmatrix}
-\dfrac{3}{\sqrt{70}} & \dfrac{2}{\sqrt{5}} & \dfrac{1}{\sqrt{14}} \\
-\dfrac{6}{\sqrt{70}} & -\dfrac{1}{\sqrt{5}} & \dfrac{2}{\sqrt{14}} \\
\dfrac{5}{\sqrt{70}} & 0 & \dfrac{3}{\sqrt{14}}
\end{bmatrix}
\]

Then the rotation matrix \(R\) rotates any vector in \(\mathbb{R}^3\) through an angle \(\alpha\) around the axis \((1, 2, 3)^T\).

\item 
\[
\text{Given system:}
\quad
\begin{aligned}
x_1 + 2x_2 + 2x_4 &= 6 \\
3x_1 + 5x_2 - x_3 + 6x_4 &= 17 \\
2x_1 + 4x_2 + 2x_4 &= 12 \\
2x_1 - 7x_3 + 11x_4 &= 7
\end{aligned}
\]

\[
\text{Augmented matrix:}
\quad
\left[
\begin{array}{rrrr|r}
1 & 2 & 0 & 2 & 6 \\
3 & 5 & -1 & 6 & 17 \\
2 & 4 & 0 & 2 & 12 \\
2 & 0 & -7 & 11 & 7
\end{array}
\right]
\]

\[
R_2 \leftarrow R_2 - 3R_1,\quad
R_3 \leftarrow R_3 - 2R_1,\quad
R_4 \leftarrow R_4 - 2R_1
\]

\[
\left[
\begin{array}{rrrr|r}
1 & 2 & 0 & 2 & 6 \\
0 & -1 & -1 & 0 & -1 \\
0 & 0 & 0 & -2 & 0 \\
0 & -4 & -7 & 7 & -5
\end{array}
\right]
\]

\[
R_2 \leftarrow -R_2
\]

\[
\left[
\begin{array}{rrrr|r}
1 & 2 & 0 & 2 & 6 \\
0 & 1 & 1 & 0 & 1 \\
0 & 0 & 0 & -2 & 0 \\
0 & -4 & -7 & 7 & -5
\end{array}
\right]
\]

\[
R_1 \leftarrow R_1 - 2R_2,\quad
R_4 \leftarrow R_4 + 4R_2
\]

\[
\left[
\begin{array}{rrrr|r}
1 & 0 & -2 & 2 & 4 \\
0 & 1 & 1 & 0 & 1 \\
0 & 0 & 0 & -2 & 0 \\
0 & 0 & -3 & 7 & -1
\end{array}
\right]
\]

\[
R_3 \leftarrow \frac{1}{-2} R_3
\]

\[
\left[
\begin{array}{rrrr|r}
1 & 0 & -2 & 2 & 4 \\
0 & 1 & 1 & 0 & 1 \\
0 & 0 & 0 & 1 & 0 \\
0 & 0 & -3 & 7 & -1
\end{array}
\right]
\]

\[
R_1 \leftarrow R_1 - 2R_3,\quad
R_4 \leftarrow R_4 - 7R_3
\]

\[
\left[
\begin{array}{rrrr|r}
1 & 0 & -2 & 0 & 4 \\
0 & 1 & 1 & 0 & 1 \\
0 & 0 & 0 & 1 & 0 \\
0 & 0 & -3 & 0 & -1
\end{array}
\right]
\]

\[
R_4 \leftarrow \frac{1}{-3} R_4
\]

\[
\left[
\begin{array}{rrrr|r}
1 & 0 & -2 & 0 & 4 \\
0 & 1 & 1 & 0 & 1 \\
0 & 0 & 0 & 1 & 0 \\
0 & 0 & 1 & 0 & \frac{1}{3}
\end{array}
\right]
\]

\[
R_1 \leftarrow R_1 + 2R_4,\quad
R_2 \leftarrow R_2 - R_4
\]

\[
\left[
\begin{array}{rrrr|r}
1 & 0 & 0 & 0 & \frac{14}{3} \\
0 & 1 & 0 & 0 & \frac{2}{3} \\
0 & 0 & 0 & 1 & 0 \\
0 & 0 & 1 & 0 & \frac{1}{3}
\end{array}
\right]
\]

\[
\text{Final solution:} \quad
x_1 = \frac{14}{3},\quad
x_2 = \frac{2}{3},\quad
x_3 = \frac{1}{3},\quad
x_4 = 0
\]

\item 

(a) Show that the kernel and image of \( T \) are vector subspaces.

Kernel: \( \ker T = \{ \mathbf{v} \in V : T(\mathbf{v}) = \mathbf{0} \} \)

  1. Zero vector: \( T(\mathbf{0}) = \mathbf{0} \Rightarrow \mathbf{0} \in \ker T \)

  2. Closed under addition: If \( \mathbf{u}, \mathbf{v} \in \ker T \), then
     \[
     T(\mathbf{u} + \mathbf{v}) = T(\mathbf{u}) + T(\mathbf{v}) = \mathbf{0} + \mathbf{0} = \mathbf{0}
     \Rightarrow \mathbf{u} + \mathbf{v} \in \ker T
     \]

  3. Closed under scalar multiplication: If \( \mathbf{v} \in \ker T \), then
     \[
     T(c\mathbf{v}) = cT(\mathbf{v}) = c\mathbf{0} = \mathbf{0}
     \Rightarrow c\mathbf{v} \in \ker T
     \]

  Hence, \( \ker T \) is a subspace of \( V \).

- Image: \( \text{Im } T = \{ T(\mathbf{v}) : \mathbf{v} \in V \} \)

  1. Zero vector: \( T(\mathbf{0}) = \mathbf{0} \Rightarrow \mathbf{0} \in \text{Im } T \)

  2. Closed under addition:
     \[
     T(\mathbf{u}) + T(\mathbf{v}) = T(\mathbf{u} + \mathbf{v}) \Rightarrow \text{Im } T \text{ is closed under addition}
     \]

  3. Closed under scalar multiplication:
     \[
     cT(\mathbf{v}) = T(c\mathbf{v}) \Rightarrow \text{Im } T \text{ is closed under scalar multiplication}
     \]

  Hence, \( \text{Im } T \) is a subspace of \( W \).

(b) \( T \) is one-to-one \( \Leftrightarrow \ker T = \{ \mathbf{0} \} \)

-  If \( T \) is one-to-one, then \( T(\mathbf{v}) = \mathbf{0} \Rightarrow \mathbf{v} = \mathbf{0} \), so \( \ker T = \{ \mathbf{0} \} \)

-  If \( \ker T = \{ \mathbf{0} \} \) and \( T(\mathbf{u}) = T(\mathbf{v}) \), then
  \[
  T(\mathbf{u} - \mathbf{v}) = T(\mathbf{u}) - T(\mathbf{v}) = \mathbf{0} \Rightarrow \mathbf{u} - \mathbf{v} \in \ker T \Rightarrow \mathbf{u} = \mathbf{v}
  \]

(c) Let \( T: \mathbb{R}^3 \to \mathbb{R}^3 \) be defined by
\[
T\left( \begin{bmatrix} x \\ y \\ z \end{bmatrix} \right) = \begin{bmatrix} x \\ 0 \\ y \end{bmatrix}
\]

- Kernel:
  \[
  T\left( \begin{bmatrix} x \\ y \\ z \end{bmatrix} \right) = \mathbf{0} \Rightarrow
  \begin{bmatrix} x \\ 0 \\ y \end{bmatrix} = \begin{bmatrix} 0 \\ 0 \\ 0 \end{bmatrix}
  \Rightarrow x = 0, y = 0
  \]
  So,
  \[
  \ker T = \left\{ \begin{bmatrix} 0 \\ 0 \\ z \end{bmatrix} : z \in \mathbb{R} \right\}
  \]
  Basis: \( \left\{ \begin{bmatrix} 0 \\ 0 \\ 1 \end{bmatrix} \right\} \), dimension = 1

- Image:
  \[
  \text{Im } T = \left\{ \begin{bmatrix} x \\ 0 \\ y \end{bmatrix} : x, y \in \mathbb{R} \right\}
  \]
  Basis: \( \left\{ \begin{bmatrix} 1 \\ 0 \\ 0 \end{bmatrix}, \begin{bmatrix} 0 \\ 0 \\ 1 \end{bmatrix} \right\} \), dimension = 2

Geometrically, the image is the \( xz \)-plane and the kernel is the \( z \)-axis.

(d) Let \( W \) be the set of all real-valued functions with continuous derivatives. Define the differentiation map
\[
D: W \to W, \quad D(f) = f'
\]

- \( D \) is linear:
  \[
  D(af + bg) = (af + bg)' = a f' + b g' = a D(f) + b D(g)
  \]

- Kernel: \( \ker D = \{ f \in W : f' = 0 \} \Rightarrow \) constant functions

- Image: All differentiable functions (i.e., all \( f \in W \)) since any differentiable function can be the derivative of some function

Now consider \( D : V_{P,n} \to V_{P,n} \), where \( V_{P,n} \) is the space of polynomials of degree at most \( n \)

- \( D(p(x)) = p'(x) \) is again a polynomial, so \( D \) is a linear transformation on \( V_{P,n} \)

- Kernel: constant polynomials ⇒ dimension = 1

- Image: polynomials of degree \( \leq n-1 \) ⇒ dimension = \( n \)


\end{enumerate}
\end{document}
