\documentclass[12pt]{article}
\usepackage{bigints}
\usepackage{graphicx}			% Use this package to include images
\usepackage{amsmath}	
\usepackage{amssymb}
\usepackage{amsfonts}
\usepackage{polynom}
\usepackage{listings}
% A library of many standard math expressions
\graphicspath{ {./Images/} }
\usepackage[margin=1in]{geometry}% Sets 1in margins. 
\newcommand{\qed}[0]{$\blacksquare$}
\usepackage{fancyhdr}			% Creates headers and footers
\usepackage{enumerate}          %These two package give custom labels to a list
\usepackage[shortlabels]{enumitem}


% Creates the header and footer. You can adjust the look and feel of these here.
\pagestyle{fancy}
\fancyhead[l]{Aditya Gupta}
\fancyhead[c]{Math 136 Homework \#1}
\fancyhead[r]{\today}
\fancyfoot[c]{\thepage}
\renewcommand{\headrulewidth}{0.2pt} %Creates a horizontal line underneath the header
\setlength{\headheight}{15pt} %Sets enough space for the header
\begin{document}
\begin{enumerate}
\item
\begin{enumerate}
    \item[\textbf{(a)}] Basis for the space of \( 3 \times 3 \) symmetric matrices:

    A \( 3 \times 3 \) symmetric matrix has the form:
    \[
    \begin{bmatrix}
    a & b & c \\
    b & d & e \\
    c & e & f
    \end{bmatrix}
    \]
    The basis consists of the following 6 matrices:
    \[
    \left\{
    \begin{bmatrix}
    1 & 0 & 0 \\ 0 & 0 & 0 \\ 0 & 0 & 0
    \end{bmatrix},
    \begin{bmatrix}
    0 & 1 & 0 \\ 1 & 0 & 0 \\ 0 & 0 & 0
    \end{bmatrix},
    \begin{bmatrix}
    0 & 0 & 1 \\ 0 & 0 & 0 \\ 1 & 0 & 0
    \end{bmatrix},
    \begin{bmatrix}
    0 & 0 & 0 \\ 0 & 1 & 0 \\ 0 & 0 & 0
    \end{bmatrix},
    \begin{bmatrix}
    0 & 0 & 0 \\ 0 & 0 & 1 \\ 0 & 1 & 0
    \end{bmatrix},
    \begin{bmatrix}
    0 & 0 & 0 \\ 0 & 0 & 0 \\ 0 & 0 & 1
    \end{bmatrix}
    \right\}
    \]

    \item[\textbf{(b)}] General basis for \( n \times n \) symmetric matrices:

    Let \( E_{ij} \) denote the matrix with a 1 in the \((i,j)\) position and 0 elsewhere.  
    Then a basis is given by:
    \[
    \left\{ E_{ii} \mid 1 \le i \le n \right\} \cup \left\{ E_{ij} + E_{ji} \mid 1 \le i < j \le n \right\}
    \]
    The total number of basis matrices is \( \frac{n(n+1)}{2} \).

    \item[\textbf{(c)}] General basis for \( n \times n \) antisymmetric matrices:

    For antisymmetric matrices, the diagonal entries are zero and \( A_{ij} = -A_{ji} \).  
    Define \( E_{ij} \) as before. Then the basis is:
    \[
    \left\{ E_{ij} - E_{ji} \mid 1 \le i < j \le n \right\}
    \]
    This yields \( \frac{n(n-1)}{2} \) antisymmetric basis matrices satisfying \( A^T = -A \).
\end{enumerate}

\item Let \( A \) be an \( m \times n \) matrix and \( B \) be an \( n \times m \) matrix.

We want to prove that \( \text{trace}(AB) = \text{trace}(BA) \).

Recall that the trace of a square matrix is the sum of its diagonal elements:
\[
\text{trace}(AB) = \sum_{i=1}^{m} (AB)_{ii}
\]

Using the definition of matrix multiplication:
\[
(AB)_{ii} = \sum_{k=1}^{n} A_{ik} B_{ki}
\]

Therefore,
\[
\text{trace}(AB) = \sum_{i=1}^{m} \sum_{k=1}^{n} A_{ik} B_{ki}
\]

Switching the order of summation:
\[
\text{trace}(AB) = \sum_{k=1}^{n} \sum_{i=1}^{m} B_{ki} A_{ik}
\]

This can be written as:
\[
\text{trace}(AB) = \sum_{k=1}^{n} (BA)_{kk} = \text{trace}(BA)
\]

Hence,
\text{trace}(AB) = \text{trace}(BA)

\item 
Let \( \mathbf{n} = \begin{bmatrix} \alpha \\ \beta \\ \gamma \end{bmatrix} \) be the normal vector to the plane \( \alpha x + \beta y + \gamma z = 0 \).

The orthogonal projection of a vector \( \mathbf{v} \in \mathbb{R}^3 \) onto the plane is given by:
\[
T(\mathbf{v}) = \mathbf{v} - \frac{\mathbf{v} \cdot \mathbf{n}}{\|\mathbf{n}\|^2} \mathbf{n}
\]

We express this as a matrix transformation:
\[
T(\mathbf{v}) = \left( I - \frac{\mathbf{n} \mathbf{n}^\top}{\|\mathbf{n}\|^2} \right) \mathbf{v}
\]

Therefore, the matrix of the transformation \( T \) with respect to the standard basis is:
\[
P = I - \frac{1}{\alpha^2 + \beta^2 + \gamma^2}
\begin{bmatrix}
\alpha^2 & \alpha \beta & \alpha \gamma \\
\alpha \beta & \beta^2 & \beta \gamma \\
\alpha \gamma & \beta \gamma & \gamma^2
\end{bmatrix}
\]


\item 
Let \( T: \mathbb{R}^3 \to \mathbb{R}^3 \) be the reflection of points through the plane \( \alpha x + \beta y + \gamma z = 0 \). This is a reflection through a plane passing through the origin with normal vector
\[
\vec{n} = \begin{bmatrix} \alpha \\ \beta \\ \gamma \end{bmatrix}.
\]
We define the unit normal vector as
\[
\hat{n} = \frac{1}{\sqrt{\alpha^2 + \beta^2 + \gamma^2}} \begin{bmatrix} \alpha \\ \beta \\ \gamma \end{bmatrix}.
\]
The matrix of the reflection transformation is given by
\[
R = I - 2 \hat{n} \hat{n}^T.
\]
Substituting \( \hat{n} \hat{n}^T \), we get
\[
\hat{n} \hat{n}^T = \frac{1}{\alpha^2 + \beta^2 + \gamma^2}
\begin{bmatrix}
\alpha \\ \beta \\ \gamma
\end{bmatrix}
\begin{bmatrix}
\alpha & \beta & \gamma
\end{bmatrix}
= \frac{1}{\alpha^2 + \beta^2 + \gamma^2}
\begin{bmatrix}
\alpha^2 & \alpha\beta & \alpha\gamma \\
\alpha\beta & \beta^2 & \beta\gamma \\
\alpha\gamma & \beta\gamma & \gamma^2
\end{bmatrix}.
\]
Thus, the reflection matrix is
\[
R = I - \frac{2}{\alpha^2 + \beta^2 + \gamma^2}
\begin{bmatrix}
\alpha^2 & \alpha\beta & \alpha\gamma \\
\alpha\beta & \beta^2 & \beta\gamma \\
\alpha\gamma & \beta\gamma & \gamma^2
\end{bmatrix}.
\]

\end{enumerate}
\end{document}
