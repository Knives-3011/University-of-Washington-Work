\documentclass[12pt]{article}
\usepackage{bigints}
\usepackage{graphicx}			% Use this package to include images
\usepackage{amsmath}	
\usepackage{amssymb}
\usepackage{amsfonts}
\usepackage{polynom}
\usepackage{listings}
% A library of many standard math expressions
\graphicspath{ {./Images/} }
\usepackage[margin=1in]{geometry}% Sets 1in margins. 
\newcommand{\qed}[0]{$\blacksquare$}
\usepackage{fancyhdr}			% Creates headers and footers
\usepackage{enumerate}          %These two package give custom labels to a list
\usepackage[shortlabels]{enumitem}


% Creates the header and footer. You can adjust the look and feel of these here.
\pagestyle{fancy}
\fancyhead[l]{Aditya Gupta}
\fancyhead[c]{Amath Homework \#3}
\fancyhead[r]{\today}
\fancyfoot[c]{\thepage}
\renewcommand{\headrulewidth}{0.2pt} %Creates a horizontal line underneath the header
\setlength{\headheight}{15pt} %Sets enough space for the header
\begin{document}

\begin{enumerate}
\item 
\begin{enumerate}

    \item Compute \( B = A A^T \)
    
    First, compute the transpose of \( A \):

    \[
    A^T =
    \begin{bmatrix}
    2 & 36 \\
    36 & 23
    \end{bmatrix}
    \]

    Now, compute \( B = A A^T \):

    \[
    B =
    \begin{bmatrix}
    2 & 36 \\
    36 & 23
    \end{bmatrix}
    \begin{bmatrix}
    2 & 36 \\
    36 & 23
    \end{bmatrix}
    \]

    Performing the matrix multiplication:

    \[
    B =
    \begin{bmatrix}
    2(2) + 36(36) & 2(36) + 36(23) \\
    36(2) + 23(36) & 36(36) + 23(23)
    \end{bmatrix}
    \]

    \[
    B =
    \begin{bmatrix}
    4 + 1296 & 72 + 828 \\
    72 + 828 & 1296 + 529
    \end{bmatrix}
    \]

    \[
    B =
    \begin{bmatrix}
    1300 & 900 \\
    900 & 1825
    \end{bmatrix}
    \]

    \item Find the eigenvalues of \( B \)

    The characteristic equation is given by:

    \[
    \det(B - \lambda I) = 0
    \]

    where \( I \) is the identity matrix.

    \[
    B - \lambda I =
    \begin{bmatrix}
    1300 - \lambda & 900 \\
    900 & 1825 - \lambda
    \end{bmatrix}
    \]

    Computing the determinant:

    \[
    (1300 - \lambda)(1825 - \lambda) - (900)(900) = 0
    \]

    Expanding:

    \[
    2372500 - 3125\lambda + \lambda^2 - 810000 = 0
    \]

    \[
    \lambda^2 - 3125\lambda + 1562500 = 0
    \]

    Solving using the quadratic formula:

    \[
    \lambda = \frac{3125 \pm \sqrt{(3125)^2 - 4(1562500)}}{2}
    \]

    \[
    \lambda = \frac{3125 \pm \sqrt{9765625 - 6250000}}{2}
    \]

    \[
    \lambda = \frac{3125 \pm \sqrt{3515625}}{2}
    \]

    \[
    \lambda = \frac{3125 \pm 1875}{2}
    \]

    \[
    \lambda_1 = \frac{3125 + 1875}{2} = 2500, \quad \lambda_2 = \frac{3125 - 1875}{2} = 625
    \]

    \item Find the eigenvectors of \( B \)

    Solve \( (B - \lambda I)v = 0 \).

    For \( \lambda_1 = 2500 \):

    \[
    \begin{bmatrix}
    -1200 & 900 \\
    900 & -675
    \end{bmatrix}
    \begin{bmatrix}
    x \\ y
    \end{bmatrix}
    =
    \begin{bmatrix}
    0 \\ 0
    \end{bmatrix}
    \]

    \[
    -1200x + 900y = 0 \Rightarrow x = \frac{900}{1200} y = \frac{3}{4} y
    \]

    Choosing \( y = 4/5 \), we get \( x = 3/5 \). So:

    \[
    v_1 =
    \begin{bmatrix}
    3/5 \\ 4/5
    \end{bmatrix}
    \]

    For \( \lambda_2 = 625 \):

    \[
    \begin{bmatrix}
    675 & 900 \\
    900 & 1200
    \end{bmatrix}
    \begin{bmatrix}
    x \\ y
    \end{bmatrix}
    =
    \begin{bmatrix}
    0 \\ 0
    \end{bmatrix}
    \]

    \[
    x = -\frac{900}{675} y = -\frac{4}{3} y
    \]

    Choosing \( y = 3/5 \), we get \( x = -4/5 \), so:

    \[
    v_2 =
    \begin{bmatrix}
    -4/5 \\ 3/5
    \end{bmatrix}
    \]
    \item  \( U \) is:

    \[
    U =
    \begin{bmatrix}
    3/5 & -4/5 \\
    4/5 & 3/5
    \end{bmatrix}
    \]

    \item Repeat with \( C = A^T A \)

    Since \( A^T A = A A^T = B \), we get:

    \[
    C = B =
    \begin{bmatrix}
    1300 & 900 \\
    900 & 1825
    \end{bmatrix}
    \]

    The eigenvalues remain \( 2500, 625 \), and the eigenvectors are chosen in the first or fourth quadrants:

    \[
    V =
    \begin{bmatrix}
    3/5 & 4/5 \\
    4/5 & -3/5
    \end{bmatrix}
    \]

    \item Construct \( \Sigma \)

    \[
    \Sigma =
    \begin{bmatrix}
    \sqrt{2500} & 0 \\
    0 & \sqrt{625}
    \end{bmatrix}
    =
    \begin{bmatrix}
    50 & 0 \\
    0 & 25
    \end{bmatrix}
    \]
\end{enumerate}

\item 
\begin{enumerate}

    \item Write out \( A = R(\pi/4) \)
    
    Substituting \( \theta = \frac{\pi}{4} \):

    \[
    A =
    \begin{bmatrix}
    \cos(\pi/4) & -\sin(\pi/4) & 0 \\
    \sin(\pi/4) & \cos(\pi/4) & 0 \\
    0 & 0 & 1
    \end{bmatrix}
    \]

    Since \( \cos(\pi/4) = \frac{\sqrt{2}}{2} \) and \( \sin(\pi/4) = \frac{\sqrt{2}}{2} \), we obtain:

    \[
    A =
    \begin{bmatrix}
    \frac{\sqrt{2}}{2} & -\frac{\sqrt{2}}{2} & 0 \\
    \frac{\sqrt{2}}{2} & \frac{\sqrt{2}}{2} & 0 \\
    0 & 0 & 1
    \end{bmatrix}
    \]

    \item Find \( A^{-1} \) using logic
    
    Since \( A \) represents a rotation by \( \frac{\pi}{4} \) (45 degrees) counterclockwise, its inverse must rotate clockwise by the same angle. This corresponds to \( \theta = -\frac{\pi}{4} \), so:

    \[
    A^{-1} = R(-\pi/4)
    \]

    Using trigonometric identities:

    \[
    \cos(-\pi/4) = \cos(\pi/4) = \frac{\sqrt{2}}{2}, \quad \sin(-\pi/4) = -\sin(\pi/4) = -\frac{\sqrt{2}}{2}
    \]

    Thus:

    \[
    A^{-1} =
    \begin{bmatrix}
    \frac{\sqrt{2}}{2} & \frac{\sqrt{2}}{2} & 0 \\
    -\frac{\sqrt{2}}{2} & \frac{\sqrt{2}}{2} & 0 \\
    0 & 0 & 1
    \end{bmatrix}
    \]

    \item Verify \( A^{-1} A = I \)

    We compute:

    \[
    A^{-1} A =
    \begin{bmatrix}
    \frac{\sqrt{2}}{2} & \frac{\sqrt{2}}{2} & 0 \\
    -\frac{\sqrt{2}}{2} & \frac{\sqrt{2}}{2} & 0 \\
    0 & 0 & 1
    \end{bmatrix}
    \begin{bmatrix}
    \frac{\sqrt{2}}{2} & -\frac{\sqrt{2}}{2} & 0 \\
    \frac{\sqrt{2}}{2} & \frac{\sqrt{2}}{2} & 0 \\
    0 & 0 & 1
    \end{bmatrix}
    \]

    Computing each entry:

    - First row:
    
      \[
      \left(\frac{\sqrt{2}}{2} \cdot \frac{\sqrt{2}}{2} + \frac{\sqrt{2}}{2} \cdot \frac{\sqrt{2}}{2} \right) = \left(\frac{2}{4} + \frac{2}{4}\right) = 1
      \]
    
      \[
      \left(\frac{\sqrt{2}}{2} \cdot -\frac{\sqrt{2}}{2} + \frac{\sqrt{2}}{2} \cdot \frac{\sqrt{2}}{2} \right) = 0
      \]

    - Second row:

      \[
      \left(-\frac{\sqrt{2}}{2} \cdot \frac{\sqrt{2}}{2} + \frac{\sqrt{2}}{2} \cdot -\frac{\sqrt{2}}{2} \right) = 0
      \]

      \[
      \left(-\frac{\sqrt{2}}{2} \cdot -\frac{\sqrt{2}}{2} + \frac{\sqrt{2}}{2} \cdot \frac{\sqrt{2}}{2} \right) = 1
      \]

    Thus:

    \[
    A^{-1} A = I =
    \begin{bmatrix}
    1 & 0 & 0 \\
    0 & 1 & 0 \\
    0 & 0 & 1
    \end{bmatrix}
    \]

    which confirms that \( A^{-1} \) is correct.

    \item
    
    - The matrix \( A \) represents rotation by \( \frac{\pi}{4} \) in the xy-plane.
    
    - The z-axis is unchanged, meaning any vector along \( (0,0,1) \) is unaffected.
    
    - This means the eigenvalue corresponding to this direction is:

    \[
    \lambda = 1
    \]

    To find the eigenvector, solve:

    \[
    A v = v
    \]

    Since the z-component is unchanged, the eigenvector is:

    \[
    v =
    \begin{bmatrix}
    0 \\
    0 \\
    1
    \end{bmatrix}
    \]

    This makes sense because rotation in the xy-plane does not affect the z-axis.

\end{enumerate}

\item 
\begin{enumerate}

    \item Formulating \( f(I_D, V_D) \)
    
    The equations are rearranged into the form:

    \[
    f(I_D, V_D) =
    \begin{bmatrix}
    f_1(I_D, V_D) \\
    f_2(I_D, V_D)
    \end{bmatrix}
    =
    \begin{bmatrix}
    I_S \left(e^{V_D / V_T} - 1 \right) - I_D \\
    \frac{V_{DD} - V_D}{R} - I_D
    \end{bmatrix}
    =
    \begin{bmatrix}
    0 \\
    0
    \end{bmatrix}
    \]

    \item Compute the Jacobian matrix
    
    The Jacobian matrix is given by:

    \[
    J(I_D, V_D) =
    \begin{bmatrix}
    \frac{\partial f_1}{\partial I_D} & \frac{\partial f_1}{\partial V_D} \\
    \frac{\partial f_2}{\partial I_D} & \frac{\partial f_2}{\partial V_D}
    \end{bmatrix}
    \]

    Computing the partial derivatives:

    \[
    \frac{\partial f_1}{\partial I_D} = -1, \quad
    \frac{\partial f_1}{\partial V_D} = \frac{I_S}{V_T} e^{V_D / V_T}
    \]

    \[
    \frac{\partial f_2}{\partial I_D} = -1, \quad
    \frac{\partial f_2}{\partial V_D} = -\frac{1}{R}
    \]

    Thus, the Jacobian matrix is:

    \[
    J(I_D, V_D) =
    \begin{bmatrix}
    -1 & \frac{I_S}{V_T} e^{V_D / V_T} \\
    -1 & -\frac{1}{R}
    \end{bmatrix}
    \]

    \item Perform one Newton-Raphson iteration
    
    \begin{enumerate}
        \item Given values
        
        \[
        I_S = 10^{-15} \text{ A}, \quad V_T = 0.025 \text{ V}, \quad V_{DD} = 1.5 \text{ V}, \quad R = 1000 \text{ Ω}
        \]

        Initial guess:

        \[
        v_0 =
        \begin{bmatrix}
        I_D \\
        V_D
        \end{bmatrix}
        =
        \begin{bmatrix}
        0 \\
        0
        \end{bmatrix}
        \]

        \item Evaluate \( f(v_0) \)

        \[
        f_1(0, 0) = I_S \left(e^{0 / 0.025} - 1 \right) - 0 = 10^{-15} (1 - 1) = 0
        \]

        \[
        f_2(0, 0) = \frac{1.5 - 0}{1000} - 0 = 0.0015
        \]

        \[
        f(v_0) =
        \begin{bmatrix}
        0 \\
        0.0015
        \end{bmatrix}
        \]

        \item Compute \( J(v_0) \)

        \[
        J(0, 0) =
        \begin{bmatrix}
        -1 & \frac{10^{-15}}{0.025} e^{0} \\
        -1 & -\frac{1}{1000}
        \end{bmatrix}
        =
        \begin{bmatrix}
        -1 & 4 \times 10^{-14} \\
        -1 & -0.001
        \end{bmatrix}
        \]

        \item Solve \( J(v_0) \Delta v = -f(v_0) \)

        \[
        \begin{bmatrix}
        -1 & 4 \times 10^{-14} \\
        -1 & -0.001
        \end{bmatrix}
        \begin{bmatrix}
        \Delta I_D \\
        \Delta V_D
        \end{bmatrix}
        =
        \begin{bmatrix}
        0 \\
        -0.0015
        \end{bmatrix}
        \]

        Using Gaussian elimination:

        \[
        \Delta I_D = 0.000815, \quad \Delta V_D = 0.682
            \]

        \item Compute \( v_1 \)

        \[
        v_1 = v_0 + \Delta v
        \]

        \[
        =
        \begin{bmatrix}
        0 \\
        0
        \end{bmatrix}
        +
        \begin{bmatrix}
        0.000815 \\
        0.682
        \end{bmatrix}
        \]

        \[
        =
        \begin{bmatrix}
        0.000815 \\
        0.682
        \end{bmatrix}
        \]

    \end{enumerate}

\end{enumerate}
\end{enumerate}

\end{document}
