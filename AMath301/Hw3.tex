\documentclass[12pt]{article}
\usepackage{bigints}
\usepackage{graphicx}			% Use this package to include images
\usepackage{amsmath}	
\usepackage{amssymb}
\usepackage{amsfonts}
\usepackage{polynom}
\usepackage{listings}
% A library of many standard math expressions
\graphicspath{ {./Images/} }
\usepackage[margin=1in]{geometry}% Sets 1in margins. 
\newcommand{\qed}[0]{$\blacksquare$}
\usepackage{fancyhdr}			% Creates headers and footers
\usepackage{enumerate}          %These two package give custom labels to a list
\usepackage[shortlabels]{enumitem}


% Creates the header and footer. You can adjust the look and feel of these here.
\pagestyle{fancy}
\fancyhead[l]{Aditya Gupta}
\fancyhead[c]{Amath Homework \#3}
\fancyhead[r]{\today}
\fancyfoot[c]{\thepage}
\renewcommand{\headrulewidth}{0.2pt} %Creates a horizontal line underneath the header
\setlength{\headheight}{15pt} %Sets enough space for the header
\begin{document}

\begin{enumerate}
\item \begin{enumerate}
    \item \( A_{2,3} \): \( 3 \) (element in the 2nd row and 3rd column of \( A \)).
    
    \item Indices of the zero entry in \( C \): \( [(1, 1)] \) (row 2, column 2 in 1-based indexing).
    
    \item \( B^T + 3C \):
    \[
    \begin{bmatrix}
    9 & 4 \\
    -4 & 1 \\
    1 & 5
    \end{bmatrix}
    \]
    
    \item \( A \circ A^T \) (Hadamard product):
    \[
    \begin{bmatrix}
    4 & 0 & 2 \\
    0 & 1 & 12 \\
    2 & 12 & 16
    \end{bmatrix}
    \]
    
    \item \( BC \):
    \[
    \begin{bmatrix}
    -17 & -4 \\
    -1 & -1
    \end{bmatrix}
    \]
    
    \item \( A^2 - I \):
    \[
    \begin{bmatrix}
    5 & -11 & -15 \\
    -3 & 12 & 15 \\
    -6 & 21 & 29
    \end{bmatrix}
    \]
\end{enumerate}

\item 
Let \( A \) be an \( m \times m \) matrix, \( B \) be an \( m \times n \) matrix, and \( q \) be an \( n \times 1 \) vector, where \( m \gg n > 1 \). The vector \( w = ABq \) can be computed in two ways:

\begin{itemize}
    \item Method 1: Compute \( AB \) (an \( m \times n \) matrix) first, followed by multiplying the result with \( q \), i.e., \( w = (AB)q \).
    \item Method 2 Compute \( Bq \) (an \( m \times 1 \) vector) first, followed by multiplying the result with \( A \), i.e., \( w = A(Bq) \).
\end{itemize}

To determine which method is computationally faster, we count the number of floating-point operations (multiplications and additions) required for each method.

\textbf{Method 1:} 
\begin{enumerate}
    \item Compute \( AB \): Multiplications require \( m \cdot n \cdot m \) operations, and additions require \( m \cdot n \cdot (m - 1) \) operations.
    \item Compute \( (AB)q \): Multiplications require \( m \cdot n \) operations, and additions require \( m \cdot (n - 1) \) operations.
\end{enumerate}
Total operations for Method 1:
\[
\text{Total} = m \cdot n \cdot m + m \cdot n \cdot (m - 1) + m \cdot n + m \cdot (n - 1)
\]

\textbf{Method 2:}
\begin{enumerate}
    \item Compute \( Bq \): Multiplications require \( n \cdot m \) operations, and additions require \( n \cdot (m - 1) \) operations.
    \item Compute \( A(Bq) \): Multiplications require \( m \cdot m \) operations, and additions require \( m \cdot (m - 1) \) operations.
\end{enumerate}
Total operations for Method 2:
\[
\text{Total} = n \cdot m + n \cdot (m - 1) + m \cdot m + m \cdot (m - 1)
\]
The difference between the two methods is:
\[
\text{Difference} = \text{Total}_1 - \text{Total}_2
\]
Substituting the expressions:
\[
\text{Difference} = [m \cdot n \cdot m + m \cdot n \cdot (m - 1) + m \cdot n + m \cdot (n - 1)]
- [n \cdot m + n \cdot (m - 1) + m \cdot m + m \cdot (m - 1)]
\]
Simplify:
\[
\text{Difference} = m \cdot n \cdot m + m \cdot n \cdot (m - 1) - m \cdot m - m \cdot (m - 1)
\]

For \( m \gg n \), the difference is dominated by \( m \cdot n \cdot m \), indicating that computing the full matrix product \( AB \) requires significantly more operations than first computing \( Bq \).

Comparison: Assume \( m = 1000 \) and \( n = 10 \):
\begin{itemize}
    \item Total operations for Method 1: \( 20,009,000 \)
    \item Total operations for Method 2: \( 2,018,990 \)
\end{itemize}

Method 2 (\( A(Bq) \)) is computationally faster as it requires significantly fewer operations. This is because computing \( Bq \) (a vector) is less costly than computing the full matrix product \( AB \).

\item 
\begin{enumerate}
    \item \( C = AB \)
If the \( p \)-th row of \( A \) is all zeros, then any element in the \( p \)-th row of \( C = AB \) is computed as:
\[
C_{p,j} = \sum_{k=1}^m A_{p,k} B_{k,j}
\]
Since \( A_{p,k} = 0 \) for all \( k \), the summation evaluates to zero:
\[
C_{p,j} = 0 \quad \text{for all } j.
\]
Thus, the \( p \)-th row of \( C \) will always be a row of zeros. This statement is true.

\item 
\( D = BA \)
If the \( p \)-th row of \( A \) is all zeros, it does not necessarily imply that \( D = BA \) will have a row of zeros. The rows of \( D \) are linear combinations of the rows of \( B \), based on the columns of \( A \). If \( A \) has a zero row, this does not affect the structure of \( D \), as \( A \)'s zero row only contributes to \( A \)'s columns, not \( B \)'s rows.
Counterexample:
Let:
\[
A = \begin{bmatrix}
1 & 0 \\
0 & 0
\end{bmatrix}, \quad B = \begin{bmatrix}
1 & 2 \\
3 & 4
\end{bmatrix}.
\]
Then:
\[
D = BA = \begin{bmatrix}
1 & 2 \\
3 & 4
\end{bmatrix} \begin{bmatrix}
1 & 0 \\
0 & 0
\end{bmatrix} = \begin{bmatrix}
1 & 0 \\
3 & 0
\end{bmatrix}.
\]
Here, no row of \( D \) is entirely zeros, despite the second row of \( A \) being all zeros. Thus, this is false
\end{enumerate} 

\item 
We start with:
\[
L =
\begin{bmatrix}
1 & 0 & 0 \\
0 & 1 & 0 \\
0 & 0 & 1
\end{bmatrix},
\quad
U =
\begin{bmatrix}
-6 & 1 & 4 \\
6 & 4 & -6 \\
18 & 7 & -15
\end{bmatrix}.
\]

First, to eliminate \( U_{21} \), we calculate the multiplier:
\[
m_{21} = \frac{U_{21}}{U_{11}} = \frac{6}{-6} = -1.
\]
Using this, we update row 2 of \( U \):
\[
\text{Row 2} \leftarrow \text{Row 2} - m_{21} \cdot \text{Row 1}.
\]
Performing the calculation:
\[
\begin{bmatrix}
-6 & 1 & 4 \\
6 & 4 & -6 \\
18 & 7 & -15
\end{bmatrix}
\longrightarrow
\begin{bmatrix}
-6 & 1 & 4 \\
0 & 5 & -2 \\
18 & 7 & -15
\end{bmatrix}.
\]
The corresponding entry in \( L \) is updated to \( m_{21} \), giving:
\[
L =
\begin{bmatrix}
1 & 0 & 0 \\
-1 & 1 & 0 \\
0 & 0 & 1
\end{bmatrix}.
\]

Next, to eliminate \( U_{31} \), we calculate:
\[
m_{31} = \frac{U_{31}}{U_{11}} = \frac{18}{-6} = -3.
\]
Update row 3 of \( U \):
\[
\text{Row 3} \leftarrow \text{Row 3} - m_{31} \cdot \text{Row 1}.
\]
This results in:
\[
\begin{bmatrix}
-6 & 1 & 4 \\
0 & 5 & -2 \\
18 & 7 & -15
\end{bmatrix}
\longrightarrow
\begin{bmatrix}
-6 & 1 & 4 \\
0 & 5 & -2 \\
0 & 10 & -3
\end{bmatrix}.
\]
Update \( L \) to include \( m_{31} \):
\[
L =
\begin{bmatrix}
1 & 0 & 0 \\
-1 & 1 & 0 \\
-3 & 0 & 1
\end{bmatrix}.
\]

To eliminate \( U_{32} \), we compute:
\[
m_{32} = \frac{U_{32}}{U_{22}} = \frac{10}{5} = 2.
\]
Update row 3 of \( U \):
\[
\text{Row 3} \leftarrow \text{Row 3} - m_{32} \cdot \text{Row 2}.
\]
The updated \( U \) is:
\[
\begin{bmatrix}
-6 & 1 & 4 \\
0 & 5 & -2 \\
0 & 10 & -3
\end{bmatrix}
\longrightarrow
\begin{bmatrix}
-6 & 1 & 4 \\
0 & 5 & -2 \\
0 & 0 & 1
\end{bmatrix}.
\]
The corresponding entry in \( L \) is updated to \( m_{32} \), resulting in:
\[
L =
\begin{bmatrix}
1 & 0 & 0 \\
-1 & 1 & 0 \\
-3 & 2 & 1
\end{bmatrix}.
\]

Thus, the \( LU \) decomposition of \( A \) is:
\[
L =
\begin{bmatrix}
1 & 0 & 0 \\
-1 & 1 & 0 \\
-3 & 2 & 1
\end{bmatrix},
\quad
U =
\begin{bmatrix}
-6 & 1 & 4 \\
0 & 5 & -2 \\
0 & 0 & 1
\end{bmatrix}.
\]
\item 
\begin{enumerate}
    \item Expand \( f(z) \):
    \[
    f(z) = \frac{1}{2} z^T A z - b^T z.
    \]
    Compute each term:
    \begin{itemize}
        \item \( z^T A z = \begin{bmatrix} x & y \end{bmatrix} \begin{bmatrix} 4 & -1 \\ -1 & 3 \end{bmatrix} \begin{bmatrix} x \\ y \end{bmatrix} = 4x^2 - xy - xy + 3y^2 = 4x^2 - 2xy + 3y^2 \).
        \item \( b^T z = \begin{bmatrix} -10 & 8 \end{bmatrix} \begin{bmatrix} x \\ y \end{bmatrix} = -10x + 8y \).
    \end{itemize}
    Combine the terms:
    \[
    f(z) = \frac{1}{2}(4x^2 - 2xy + 3y^2) - (-10x + 8y) = 2x^2 - xy + \frac{3}{2}y^2 + 10x - 8y.
    \]

    \item Compute the gradient \( \nabla f(z) \):
    \[
    \nabla f(z) = \begin{bmatrix} \frac{\partial f}{\partial x} \\ \frac{\partial f}{\partial y} \end{bmatrix}.
    \]
    \begin{itemize}
        \item \( \frac{\partial f}{\partial x} = \frac{\partial}{\partial x} \left( 2x^2 - xy + \frac{3}{2}y^2 + 10x - 8y \right) = 4x - y + 10 \).
        \item \( \frac{\partial f}{\partial y} = \frac{\partial}{\partial y} \left( 2x^2 - xy + \frac{3}{2}y^2 + 10x - 8y \right) = -x + 3y - 8 \).
    \end{itemize}
    Thus:
    \[
    \nabla f(z) = \begin{bmatrix} 4x - y + 10 \\ -x + 3y - 8 \end{bmatrix}.
    \]

    \item Perform two iterations of Gradient Descent starting at \( z_0 = \begin{bmatrix} 0 \\ 0 \end{bmatrix} \) with \( \tau = \frac{1}{4} \):
    \begin{enumerate}
        \item Compute \( z_1 \):
        \[
        z_1 = z_0 - \tau \nabla f(z_0).
        \]
        Substitute \( z_0 = \begin{bmatrix} 0 \\ 0 \end{bmatrix} \):
        \[
        \nabla f(z_0) = \begin{bmatrix} 4(0) - 0 + 10 \\ -(0) + 3(0) - 8 \end{bmatrix} = \begin{bmatrix} 10 \\ -8 \end{bmatrix}.
        \]
        \[
        z_1 = \begin{bmatrix} 0 \\ 0 \end{bmatrix} - \frac{1}{4} \begin{bmatrix} 10 \\ -8 \end{bmatrix} = \begin{bmatrix} -\frac{5}{2} \\ 2 \end{bmatrix}.
        \]

        \item Compute \( z_2 \):
        \[
        z_2 = z_1 - \tau \nabla f(z_1).
        \]
        Substitute \( z_1 = \begin{bmatrix} -\frac{5}{2} \\ 2 \end{bmatrix} \):
        \[
        \nabla f(z_1) = \begin{bmatrix} 4(-\frac{5}{2}) - 2 + 10 \\ -(-\frac{5}{2}) + 3(2) - 8 \end{bmatrix} = \begin{bmatrix} -10 - 2 + 10 \\ \frac{5}{2} + 6 - 8 \end{bmatrix} = \begin{bmatrix} -2 \\ \frac{5}{2} \end{bmatrix}.
        \]
        \[
        z_2 = \begin{bmatrix} -\frac{5}{2} \\ 2 \end{bmatrix} - \frac{1}{4} \begin{bmatrix} -2 \\ \frac{5}{2} \end{bmatrix} = \begin{bmatrix} -\frac{5}{2} + \frac{1}{2} \\ 2 - \frac{5}{8} \end{bmatrix} = \begin{bmatrix} -2 \\ \frac{11}{8} \end{bmatrix}.
        \]
    \end{enumerate}
    Thus:
    \[
    z_1 = \begin{bmatrix} -\frac{5}{2} \\ 2 \end{bmatrix}, \quad z_2 = \begin{bmatrix} -2 \\ \frac{11}{8} \end{bmatrix}.
    \]
\end{enumerate}


\end{enumerate}

\end{document}
