\documentclass[12pt]{article}
\usepackage{bigints}
\usepackage{graphicx}			% Use this package to include images
\usepackage{amsmath}	
\usepackage{amssymb}
\usepackage{amsfonts}
\usepackage{polynom}
% A library of many standard math expressions
\graphicspath{ {./Images/} }
\usepackage[margin=1in]{geometry}% Sets 1in margins. 
\newcommand{\qed}[0]{$\blacksquare$}
\usepackage{fancyhdr}			% Creates headers and footers
\usepackage{enumerate}          %These two package give custom labels to a list
\usepackage[shortlabels]{enumitem}


% Creates the header and footer. You can adjust the look and feel of these here.
\pagestyle{fancy}
\fancyhead[l]{Aditya Gupta}
\fancyhead[c]{Math 134 Homework \#11}
\fancyhead[r]{\today}
\fancyfoot[c]{\thepage}
\renewcommand{\headrulewidth}{0.2pt} %Creates a horizontal line underneath the header
\setlength{\headheight}{15pt} %Sets enough space for the header
\begin{document}

\begin{enumerate}
\item [30.]
We solve the differential equation:
\[
y' + ry = 0,
\]
where \( r \) is constant.

The general solution is found as follows:
The equation is separable:
\[
\frac{dy}{dx} = -ry.
\]
Separating variables:
\[
\frac{1}{y} \, dy = -r \, dx.
\]
Integrating both sides:
\[
\ln |y| = -rx + C.
\]
Exponentiating:
\[
y = Ce^{-rx}.
\]
Thus, the general solution is:
\[
y(x) = Ce^{-rx},
\]
where \( C \) is an arbitrary constant.

Now we address the individual parts of the problem:

\begin{enumerate}
    \item[(a)] If \( y(a) = 0 \) for some \( a \geq 0 \), then \( y(x) = 0 \) for all \( x \)\\
    From the general solution:
    \[
    y(x) = Ce^{-rx}.
    \]
    If \( y(a) = 0 \), then:
    \[
    C e^{-ra} = 0.
    \]
    Since \( e^{-ra} \neq 0 \) for any \( r \) and \( a \), it must be that \( C = 0 \). If \( C = 0 \), then:
    \[
    y(x) = 0 \quad \text{for all } x.
    \]
    This shows that \( y(x) \) is either identically zero (\( y(x) = 0 \)) or never zero.

    \item[(b)] If \( r < 0 \), all nonzero solutions are unbounded\\
    For \( r < 0 \), the solution is:
    \[
    y(x) = Ce^{-rx}.
    \]
    Since \( r < 0 \), \( -r > 0 \), so \( e^{-rx}\) grows exponentially as \( x \to \infty \). For \( C \neq 0 \), \( y(x) \) becomes unbounded as \( x \to \infty \).

    \item[(c)] If \( r > 0 \), all solutions tend to 0 as \( x \to \infty \)\\
    For \( r > 0 \), the solution is:
    \[
    y(x) = Ce^{-rx}.
    \]
    Since \( r > 0 \), \( e^{-rx} \to 0 \) as \( x \to \infty \). Thus, for any \( C \), \( y(x) \to 0 \) as \( x \to \infty \).

    \item[(d)] Solutions if \( r = 0 \)\\
    If \( r = 0 \), the equation becomes:
    \[
    y' = 0.
    \]
    Integrating:
    \[
    y = C,
    \]
    where \( C \) is a constant. Thus, the solution is a constant function.
\end{enumerate}

\item [31. ]
\begin{enumerate}
    \item Show that if \( y_1 \) and \( y_2 \) are solutions, then \( u = y_1 + y_2 \) is also a solution.
        If \( y_1 \) is a solution, then:
        \[
        y_1' + p(x)y_1 = 0
        \]
        If \( y_2 \) is a solution, then:
        \[
        y_2' + p(x)y_2 = 0
        \]
        Define \( u = y_1 + y_2 \). Then:
        \[
        u' = y_1' + y_2'
        \]
        Substitute \( u \) and \( u' \) into the original differential equation:
        \[
        u' + p(x)u = (y_1' + y_2') + p(x)(y_1 + y_2)
        \]
        Simplify:
        \[
        u' + p(x)u = (y_1' + p(x)y_1) + (y_2' + p(x)y_2)
        \]
        Since \( y_1' + p(x)y_1 = 0 \) and \( y_2' + p(x)y_2 = 0 \), we have:
        \[
        u' + p(x)u = 0 + 0 = 0
        \]
        Therefore, \( u = y_1 + y_2 \) is a solution.

    \item Show that if \( y \) is a solution and \( C \) is a constant, then \( u = Cy \) is also a solution.

        If \( y \) is a solution, then:
        \[
        y' + p(x)y = 0
        \]
        Define \( u = Cy \), where \( C \) is a constant. Then:
        \[
        u' = C y'
        \]
        Substitute \( u \) and \( u' \) into the original differential equation:
        \[
        u' + p(x)u = C y' + p(x)(C y)
        \]
        Factor out \( C \):
        \[
        u' + p(x)u = C (y' + p(x)y)
        \]
        Since \( y' + p(x)y = 0 \), we have:
        \[
        u' + p(x)u = C \cdot 0 = 0
        \]
        Therefore, \( u = Cy \) is a solution.
\end{enumerate}

\item [34. ] Show that if \( y_1 \) and \( y_2 \) are solutions of \( y' + p(x)y = q(x) \), then \( y = y_1 - y_2 \) is a solution of \( y' + p(x)y = 0 \).
    \begin{proof}
        If \( y_1 \) is a solution of \( y' + p(x)y = q(x) \), then:
        \[
        y_1' + p(x)y_1 = q(x)
        \]
        Similarly, if \( y_2 \) is a solution of \( y' + p(x)y = q(x) \), then:
        \[
        y_2' + p(x)y_2 = q(x)
        \]
        Define \( y = y_1 - y_2 \). Then:
        \[
        y' = y_1' - y_2'
        \]
        Substitute \( y \) and \( y' \) into the differential equation \( y' + p(x)y \):
        \[
        y' + p(x)y = (y_1' - y_2') + p(x)(y_1 - y_2)
        \]
        Simplify:
        \[
        y' + p(x)y = (y_1' + p(x)y_1) - (y_2' + p(x)y_2)
        \]
        Since \( y_1' + p(x)y_1 = q(x) \) and \( y_2' + p(x)y_2 = q(x) \), we have:
        \[
        y' + p(x)y = q(x) - q(x) = 0
        \]
        Therefore, \( y = y_1 - y_2 \) is a solution of \( y' + p(x)y = 0 \).
    \end{proof}
\item [23. ] 
\begin{enumerate}

\item We start with the differential equation:
    \[
    \frac{m \, dv}{dt} = -\alpha v - \beta v^2.
    \]

    Rearranging, we get:
    \[
    \frac{dv}{\alpha v + \beta v^2} = -\frac{1}{m} \, dt.
    \]

    Factorizing the denominator:
    \[
    \alpha v + \beta v^2 = v(\alpha + \beta v).
    \]

    The equation becomes:
    \[
    \int \frac{1}{v(\alpha + \beta v)} \, dv = -\frac{1}{m} \int dt.
    \]

    To solve the integral on the left, decompose the fraction:
    \[
    \frac{1}{v(\alpha + \beta v)} = \frac{1}{\alpha} \left( \frac{1}{v} - \frac{\beta}{\alpha + \beta v} \right).
    \]

    Thus, the integral becomes:
    \[
    \int \frac{1}{v(\alpha + \beta v)} \, dv = \frac{1}{\alpha} \int \frac{1}{v} \, dv - \frac{\beta}{\alpha} \int \frac{1}{\alpha + \beta v} \, dv.
    \]

    Solve each term:
    \[
    \int \frac{1}{v} \, dv = \ln|v|, \quad \int \frac{1}{\alpha + \beta v} \, dv = \frac{1}{\beta} \ln|\alpha + \beta v|.
    \]

    Substituting back:
    \[
    \int \frac{1}{v(\alpha + \beta v)} \, dv = \frac{1}{\alpha} \ln|v| - \frac{1}{\alpha} \ln|\alpha + \beta v| + C.
    \]

    Combining the logarithms:
    \[
    \frac{1}{\alpha} \ln\left(\frac{v}{\alpha + \beta v}\right) = -\frac{1}{m}t + C_1.
    \]

    Multiply through by \( \alpha \):
    \[
    \ln\left(\frac{v}{\alpha + \beta v}\right) = -\frac{\alpha}{m}t + C_2,
    \]
    where \( C_2 = \alpha C_1 \).

    Exponentiate both sides:
    \[
    \frac{v}{\alpha + \beta v} = Ce^{-\frac{\alpha}{m}t},
    \]
    where \( C = e^{C_2} \).

    Rearrange to isolate \( v \):
    \[
    \alpha + \beta v = \frac{\alpha}{v} \implies v \left(Ce^{-\frac{\alpha}{m}t} - \beta\right) = \alpha.
    \]

    Solving for \( v \):
    \[
    v = \frac{\alpha}{Ce^{\frac{\alpha}{m}t} - \beta}.
    \]

\item 
    Starting with the general solution:
    \[
    v(t) = \frac{\alpha}{C e^{\frac{\alpha}{m}t} - \beta}.
    \]
    
    Apply the initial condition \(v(0) = v_0\):
    
    Substitute \(t = 0\) and \(v(0) = v_0\) into the solution:
    \[
    v_0 = \frac{\alpha}{C - \beta}.
    \]
    
    Rearrange to solve for \(C\):
    \[
    C - \beta = \frac{\alpha}{v_0}.
    \]
    
    Thus:
    \[
    C = \frac{\alpha}{v_0} + \beta.
    \]
    
    Substituting \(C\) back into \(v(t)\):
    
    Substitute \(C = \frac{\alpha}{v_0} + \beta\) into the general solution:
    \[
    v(t) = \frac{\alpha}{\left(\frac{\alpha}{v_0} + \beta\right)e^{\frac{\alpha}{m}t} - \beta}.
    \]
    
    Multiply the numerator and denominator by \(v_0\) to simplify:
    \[
    v(t) = \frac{\alpha v_0}{\alpha e^{\frac{\alpha}{m}t} + \beta v_0 (e^{\frac{\alpha}{m}t} - 1)}.
    \]

    \item
    The denominator is:
    \[
    \alpha e^{\frac{\alpha}{m}t} + \beta v_0 (e^{\frac{\alpha}{m}t} - 1).
    \]

    Factor \(e^{\frac{\alpha}{m}t}\) from the terms involving the exponential:
    \[
    \alpha e^{\frac{\alpha}{m}t} + \beta v_0 (e^{\frac{\alpha}{m}t} - 1) = e^{\frac{\alpha}{m}t} \left( \alpha + \beta v_0 \right) - \beta v_0.
    \]

    As \(t \to \infty\), the term \(-\beta v_0\) becomes negligible compared to the \(e^{\frac{\alpha}{m}t}\)-dependent term. Thus, the denominator approximates:
    \[
    \alpha e^{\frac{\alpha}{m}t} + \beta v_0 (e^{\frac{\alpha}{m}t} - 1) \sim e^{\frac{\alpha}{m}t} (\alpha + \beta v_0).
    \]

    For large \(t\), substitute the approximation of the denominator into \(v(t)\):
    \[
    v(t) \sim \frac{a v_0}{e^{\frac{\alpha}{m}t} (\alpha + \beta v_0)}.
    \]

    Factor \(e^{\frac{\alpha}{m}t}\) from the denominator:
    \[
    v(t) \sim \frac{\alpha v_0}{(\alpha + \beta v_0)e^{\frac{\alpha}{m}t}}.
    \]

    Since \(e^{\frac{\alpha}{m}t} \to \infty\) as \(t \to \infty\), the fraction:
    \[
    \frac{1}{e^{\frac{\alpha}{m}t}} \to 0.
    \]

    Thus:
    \[
    v(t) \to 0 \quad \text{as } t \to \infty.
    \]

\end{enumerate}
\item [24. ]

\begin{enumerate}
    \item Expression for $t$ in terms of $v$, $v_0$, and $v_c$
    
    The net force acting on the parachutist is:
    \[
    m \frac{dv}{dt} = mg - \beta v^2.
    \]

    Rewriting:
    \[
    \frac{dv}{dt} = g \left(1 - \frac{v^2}{v_c^2}\right),
    \]
    where $v_c = \sqrt{\frac{mg}{\beta}}$.

    Separating variables:
    \[
    \frac{dv}{1 - \frac{v^2}{v_c^2}} = g \, dt.
    \]

    Using partial fraction decomposition:
    \[
    \frac{1}{1 - \frac{v^2}{v_c^2}} = \frac{v_c}{2(v_c - v)} + \frac{v_c}{2(v_c + v)}.
    \]

    Integrating both sides:
    \[
    \int \left( \frac{1}{v_c - v} + \frac{1}{v_c + v} \right) dv = \int \frac{2g}{v_c} dt.
    \]

    The left-hand side becomes:
    \[
    \frac{1}{2v_c} \ln \left| \frac{v_c + v}{v_c - v} \right|.
    \]

    The right-hand side:
    \[
    \frac{2g}{v_c} t + C.
    \]

    At $t = 0$, $v = v_0$, giving:
    \[
    C = \frac{1}{2v_c} \ln \left| \frac{v_c + v_0}{v_c - v_0} \right|.
    \]

    Substituting $C$:
    \[
    \frac{1}{2v_c} \ln \left| \frac{v_c + v}{v_c - v} \right| = \frac{2g}{v_c} t + \frac{1}{2v_c} \ln \left| \frac{v_c + v_0}{v_c - v_0} \right|.
    \]

    Rearranging for $t$:
    \[
    t = \frac{v_c}{2g} \ln \left| \frac{(v_c + v)(v_c - v_0)}{(v_c - v)(v_c + v_0)} \right|.
    \]

    \item Velocity $v(t)$ as a function of time

    Starting from:
    \[
    t = \frac{v_c}{2g} \ln \left( \frac{(v_c + v)(v_c - v_0)}{(v_c - v)(v_c + v_0)} \right).
    \]

    Multiply through by $\frac{2g}{v_c}$:
    \[
    \ln \left( \frac{(v_c + v)(v_c - v_0)}{(v_c - v)(v_c + v_0)} \right) = \frac{2g}{v_c} t.
    \]

    Exponentiate both sides:
    \[
    \frac{(v_c + v)(v_c - v_0)}{(v_c - v)(v_c + v_0)} = e^{\frac{2g}{v_c} t}.
    \]

    Cross-multiply:
    \[
    (v_c + v)(v_c - v_0) = e^{\frac{2g}{v_c} t} (v_c - v)(v_c + v_0).
    \]

    Expand and rearrange terms:
    \[
    v = \frac{e^{\frac{2g}{v_c} t} (v_c^2 + v_0 v_c) - (v_c^2 - v_0 v_c)}{e^{\frac{2g}{v_c} t} (-v_c - v_0) + (v_c - v_0)}.
    \]

    Simplified form:
    \[
    v(t) = v_c \frac{1 - e^{-\frac{2g}{v_c}t}}{1 + e^{-\frac{2g}{v_c}t}}.
    \]

    \item Acceleration $a(t)$ as a function of time

    The acceleration is:
    \[
    a(t) = g \left( 1 - \frac{v^2}{v_c^2} \right).
    \]

    Substituting $v(t)$:
    \[
    v(t)^2 = v_c^2 \frac{(1 - e^{-\frac{2g}{v_c}t})^2}{(1 + e^{-\frac{2g}{v_c}t})^2}.
    \]

    So:
    \[
    a(t) = g \frac{4e^{-\frac{2g}{v_c}t}}{(1 + e^{-\frac{2g}{v_c}t})^2}.
    \]

    As $t \to \infty$, $e^{-\frac{2g}{v_c}t} \to 0$, and:
    \[
    a(t) \to 0.
    \]

    \item Show that $v(t) \to v_c$ as $t \to \infty$

    From:
    \[
    v(t) = v_c \frac{1 - e^{-\frac{2g}{v_c}t}}{1 + e^{-\frac{2g}{v_c}t}},
    \]

    as $t \to \infty$, $e^{-\frac{2g}{v_c}t} \to 0$, so:
    \[
    v(t) \to v_c \frac{1 - 0}{1 + 0} = v_c.
    \]

    Thus:
    \[
    \lim_{t \to \infty} v(t) = v_c.
    \]

\end{enumerate}

\item [27.] 
\[
F_{\text{net}} = F_g - F_r = m \frac{dv}{dt},
\] 
which gives the differential equation 
\[
100 \frac{dv}{dt} = 980 - 2v,
\] 
or equivalently, 
\[
\frac{dv}{dt} = 9.8 - 0.02v.
\] 
Separating variables, we get 
\[
\frac{dv}{9.8 - 0.02v} = dt,
\] 
and integrating both sides yields 
\[
\int \frac{1}{9.8 - 0.02v} dv = \int dt,
\] 
which simplifies to 
\[
-50 \ln|9.8 - 0.02v| = t + C.
\] 
Solving for \( v \), we find 
\[
v = \frac{9.8 - e^{-\frac{t + C}{50}}}{0.02}.
\] 
Using the initial condition \( v(0) = 0 \), we determine 
\[
C = -50 \ln(9.8).
\] 
Substituting \( C \) into the velocity equation, we have 
\[
v = \frac{9.8 - e^{-\frac{t - 50\ln(9.8)}{50}}}{0.02}.
\] 
At \( t = 10 \), substituting into this equation gives 
\[
v(10) \approx 88.82 \, \text{m/s}.
\] 
Therefore, the velocity of the package at the instant the parachute opens is approximately \( 88.82 \, \text{m/s} \).

\[
F_g = mg = 980 \, \text{N}, \quad F_r = 4v^2, \quad \text{and} \quad F_{\text{net}} = F_g - F_r = m \frac{dv}{dt}.
\]

\[
100 \frac{dv}{dt} = 980 - 4v^2,
\]
which simplifies to:
\[
\frac{dv}{dt} = 9.8 - 0.04v^2.
\]
Separating variables:
\[
\frac{dv}{9.8 - 0.04v^2} = dt.
\]
Rewriting the denominator:
\[
9.8 - 0.04v^2 = 9.8 \left(1 - \frac{v^2}{245}\right).
\]
Thus:
\[
\int \frac{dv}{9.8 \left(1 - \frac{v^2}{245}\right)} = \int dt.
\]
Factoring out \( 9.8 \) and recognizing the form of the integral:
\[
\frac{1}{9.8} \int \frac{1}{1 - \frac{v^2}{245}} dv = \int dt.
\]
The integral on the left evaluates to:
\[
\frac{1}{9.8} \cdot \frac{5^{3/2}}{14} \left( \ln|v + 7\sqrt{5}| - \ln|v - 7\sqrt{5}| \right) = t + C.
\]
Simplify:
\[
\frac{5^{3/2}}{14} \left( \ln|v + 7\sqrt{5}| - \ln|v - 7\sqrt{5}| \right) = 9.8 (t + C).
\]
At \( t = 0 \), the velocity is \( v(0) = 88.82 \, \text{m/s} \), so:
\[
\frac{5^{3/2}}{14} \left( \ln|88.82 + 7\sqrt{5}| - \ln|88.82 - 7\sqrt{5}| \right) = 9.8 C.
\]
Calculate the constant:
\[
C = \frac{1}{9.8} \cdot \frac{5^{3/2}}{14} \left( \ln|88.82 + 7\sqrt{5}| - \ln|88.82 - 7\sqrt{5}| \right).
\]
Numerically, \( C \approx 0.284 \).

The equation for \( v(t) \) becomes:
\[
\frac{5^{3/2}}{14} \left( \ln|v + 7\sqrt{5}| - \ln|v - 7\sqrt{5}| \right) = 9.8t + 9.8 \cdot 0.284.
\]
Exponentiate both sides:
\[
\frac{v + 7\sqrt{5}}{v - 7\sqrt{5}} = e^{\frac{14}{5^{3/2}} (t + 0.284)}.
\]
Reorganize:
\[
v + 7\sqrt{5} = (v - 7\sqrt{5}) \cdot e^{\frac{14}{5^{3/2}} (t + 0.284)}.
\]
Simplify to solve for \( v(t) \):
\[
v(t) = \frac{-7\sqrt{5} \left(e^{\frac{14}{5^{3/2}} (t + 0.284)} + 1\right)}{1 - e^{\frac{14}{5^{3/2}} (t + 0.284)}}.
\]
Numerically, the equation becomes:
\[
v(t) = \frac{-7\sqrt{5} (1.427 \cdot e^{\frac{14}{5^{3/2}}t} + 1)}{1 - 1.427 \cdot e^{\frac{14}{5^{3/2}}t}}.
\]

To find the terminal velocity of the package, we consider the condition when the net acceleration becomes zero. This occurs when the gravitational force is balanced by the air resistance:
\[
F_g = F_r.
\]
Given:
\[
F_g = 980 \, \text{N}, \quad F_r = 4v^2.
\]
At terminal velocity:
\[
980 = 4v^2.
\]
Solve for \( v \):
\[
v^2 = \frac{980}{4} = 245,
\]
\[
v = \sqrt{245}.
\]
\[
v \approx 15.65 \, \text{m/s}.
\]
Thus, the terminal velocity of the package is approximately \( 15.65 \, \text{m/s} \).


\item [Cooling]

\begin{enumerate}
    \item 
    To find the time at which the coffee should cool to a temperature \(T(u)\), such that after adding cream, the resulting temperature is \(110^\circ \text{F}\), we solve:
    \[
    T_{\text{mix}} = \frac{12 \cdot T(u) + 2 \cdot 40}{12 + 2} = 110.
    \]
    Simplify to find \(T(u)\):
    \[
    12 \cdot T(u) + 80 = 1540 \quad \Rightarrow \quad T(u) = 121.67^\circ \text{F}.
    \]
    Using Newton's law of cooling:
    \[
    T(u) = T_{\text{room}} + (T_0 - T_{\text{room}})e^{-ku},
    \]
    where \(T_{\text{room}} = 70^\circ \text{F}\), \(T_0 = 180^\circ \text{F}\), and \(T(u) = 121.67^\circ \text{F}\). Substituting these values:
    \[
    121.67 = 70 + 110e^{-ku} \quad \Rightarrow \quad e^{-ku} = \frac{51.67}{110}.
    \]
    Solving for \(u\) using \(k = -\frac{1}{20} \ln\left(\frac{40}{110}\right)\):
    \[
    u = \frac{20 \ln\left(\frac{51.67}{110}\right)}{\ln\left(\frac{40}{110}\right)} \approx 14.94 \, \text{minutes}.
    \]

    \item  If cream is added immediately, the initial temperature of the coffee-cream mixture becomes:
    \[
    T_{\text{mix}} = \frac{12 \cdot 180 + 2 \cdot 40}{12 + 2} = 160^\circ \text{F}.
    \]
    Using Newton's law of cooling:
    \[
    T(t) = T_{\text{room}} + (T_{\text{mix}} - T_{\text{room}})e^{-kt},
    \]
    where \(T_{\text{mix}} = 160^\circ \text{F}\) and \(T_{\text{room}} = 70^\circ \text{F}\). Solving for \(t\) when \(T(t) = 110^\circ \text{F}\):
    \[
    110 = 70 + (160 - 70)e^{-kt} \quad \Rightarrow \quad e^{-kt} = \frac{40}{90}.
    \]
    Using \(k = -\frac{1}{20} \ln\left(\frac{40}{110}\right)\), the time is:
    \[
    t = \frac{20 \ln\left(\frac{40}{90}\right)}{\ln\left(\frac{40}{110}\right)} \approx 16.03 \, \text{minutes}.
    \]

    \item If cream is added after 5 minutes, the temperature of the coffee at \(t = 5\) is:
    \[
    T(5) = T_{\text{room}} + (T_0 - T_{\text{room}})e^{-5k} = 70 + 110e^{-5k}.
    \]
    After adding cream, the new temperature is:
    \[
    T_{\text{mix}} = \frac{12 \cdot T(5) + 2 \cdot 40}{14} = \frac{12 \cdot \left(70 + 110e^{-5k}\right) + 80}{14}.
    \]
    The coffee-cream mixture cools to \(110^\circ \text{F}\) according to:
    \[
    110 = 70 + (T_{\text{mix}} - 70)e^{-k(t - 5)}.
    \]
    Solving for \(t - 5\):
    \[
    e^{-k(t - 5)} = \frac{40}{T_{\text{mix}} - 70}.
    \]
    Using \(k = -\frac{1}{20} \ln\left(\frac{40}{110}\right)\), the total time is:
    \[
    t = 5 + \frac{20 \ln\left(\frac{40}{T_{\text{mix}} - 70}\right)}{\ln\left(\frac{40}{110}\right)} \approx 15.76 \, \text{minutes}.
    \]

    \item  The total cooling time as a function of \(u\), the time at which cream is added, is:
    \[
    t_{\text{total}}(u) = u + 20 - \frac{20}{\ln\frac{4}{11}} \ln\left(\frac{6}{7}e^{\frac{\ln\frac{4}{11}u}{20}} - \frac{3}{77}\right).
    \]
    \[
t'_{\text{total}}(u) = 1 - \frac{20}{\ln\frac{4}{11}} \cdot \frac{1}{\frac{6}{7}e^{\frac{\ln\frac{4}{11}u}{20}} - \frac{3}{77}} \cdot \frac{d}{du}\left(\frac{6}{7}e^{\frac{\ln\frac{4}{11}u}{20}} - \frac{3}{77}\right).
\]

\[
\frac{d}{du}\left(\frac{6}{7}e^{\frac{\ln\frac{4}{11}u}{20}} - \frac{3}{77}\right) = \frac{6}{7} \cdot e^{\frac{\ln\frac{4}{11}u}{20}} \cdot \frac{\ln\frac{4}{11}}{20}.
\]
Substitute this back into \(t'_{\text{total}}(u)\):
\[
t'_{\text{total}}(u) = 1 - \frac{20}{\ln\frac{4}{11}} \cdot \frac{\frac{6}{7}e^{\frac{\ln\frac{4}{11}u}{20}} \cdot \frac{\ln\frac{4}{11}}{20}}{\frac{6}{7}e^{\frac{\ln\frac{4}{11}u}{20}} - \frac{3}{77}}.
\]
Simplify:
\[
t'_{\text{total}}(u) = 1 - \frac{\frac{6}{7}e^{\frac{\ln\frac{4}{11}u}{20}}}{\frac{6}{7}e^{\frac{\ln\frac{4}{11}u}{20}} - \frac{3}{77}}.
\]
Observing the terms:

- The numerator \(\frac{6}{7}e^{\frac{\ln\frac{4}{11}u}{20}}\) is positive.

- The denominator \(\frac{6}{7}e^{\frac{\ln\frac{4}{11}u}{20}} - \frac{3}{77}\) is also positive for \(u \in [0, 14.93]\), as \(-\frac{3}{77}\) does not dominate the leading term.

Thus, the fraction is always less than 1, and \(t'_{\text{total}}(u) < 0\).

\[
t'_{\text{total}}(u) < 0 \quad \text{for all } u \in [0, 14.93].
\]
    By showing that \(t'_{\text{total}}(u) < 0\) for \(u \in [0, 14.93]\), we conclude that the time function is minimized at \(u = 14.93\) for the existing bounds, meaning the cream should be added at the very end.
\end{enumerate}



\end{enumerate}

\end{document}
