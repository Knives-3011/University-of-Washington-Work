\documentclass[12pt]{article}

\usepackage{graphicx}			% Use this package to include images
\usepackage{amsmath}	
\usepackage{amsfonts}
\usepackage{polynom}
% A library of many standard math expressions
\graphicspath{ {./Images/} }
\usepackage[margin=1in]{geometry}% Sets 1in margins. 
\usepackage{fancyhdr}			% Creates headers and footers
\usepackage{enumerate}          %These two package give custom labels to a list
\usepackage[shortlabels]{enumitem}


% Creates the header and footer. You can adjust the look and feel of these here.
\pagestyle{fancy}
\fancyhead[l]{Aditya Gupta}
\fancyhead[c]{Math 134 Homework \#4}
\fancyhead[r]{\today}
\fancyfoot[c]{\thepage}
\renewcommand{\headrulewidth}{0.2pt} %Creates a horizontal line underneath the header
\setlength{\headheight}{15pt} %Sets enough space for the header



\begin{document} 
\begin{enumerate}[start=1,label={\bfseries. },leftmargin=1in]
\item [21. ]

    \[
    F(x) = \int_0^x \left[ t \int_1^t f(u) \, du \right] dt.
    \]
    
    \begin{enumerate}
    \item
    $F'(x) = x \int_1^x f(u) \, du$
    \item 
    $F'(1) = 0$
    \item
    $F''(x) = x f(x) + \int_1^x f(u) \, du$
    \item 
    $F''(1) = f(1)$
    \end{enumerate} 

\item [36. ]

\begin{enumerate}
    \item $F'(x) = x \int_1^x f(u)du$
    \item $F'(1) = 0$
    \item $F''(x) = \int_1^x f(u)du + x\cdot f(x)$
    \item $F''(1) = f(1)$
\end{enumerate}

\item [2. ]
Let $f(x) = 4x-x^2$ and the line as $g(x)=ax$. 

To find the area under curves, we must first find the intersection of both curves:

\[
ax= 4x-x^2
\]
\[
x^2 - 4x + ax =0
\]
\[
x(x+a-4) = 0
\]

Thus:
\[
x = 0 \quad or \quad x=4-a
\]

\[
\int_0^{4-a} ((4-a)x -x^2)dx = \frac{1}{2}\int_0^4 4x-x^2dx
\]
\[
\left(\frac{(4-a)x^2}{2} - \frac{x^3}{3}\right) \Bigg|_0^{4-a} = \frac{1}{2}\left(2x^2 - \frac{x^3}{3}\right) \Bigg|_0^{4}
\]
\[
\frac{(4-a)^3}{2} - \frac{(4-a)^3}{3} = \frac{1}{2} \left(2(4^2) - \frac{4^3}{3}\right) - 0
\]
\[
\frac{(4-a)^3}{6} = 5.\overline{33}
\]
\[
(4-a)^3 = 32
\]
\[
a \approx 0.825
\]

\item [16]
\begin{enumerate}
    \item This statement is true as:
\[
(f + g)_{\text{avg}} = \frac{1}{b - a} \int_a^b (f(x) + g(x)) \, dx 
\]
\[
= \frac{1}{b - a} \int_a^b f(x) \, dx + \frac{1}{b - a} \int_a^b g(x) \, dx = f_{\text{avg}} + g_{\text{avg}}
\]
\item This statement is true as \(\alpha\) is a constant, we can factor it out of the integral:

\[
(\alpha f)_{\text{avg}} = \frac{1}{b - a} \int_a^b \alpha f(x) \, dx 
\]
\[
= \alpha \cdot \frac{1}{b - a} \int_a^b f(x) \, dx = \alpha f_{\text{avg}}
\]

\item This statement is false as:
\[
(fg)_{avg} = \frac{1}{b - a} \int_a^b f(x) g(x)dx
\]
\[
f_{avg} = \frac{1}{b - a} \int_a^b f(x)dx
\]
\[
g_{avg} = \frac{1}{b - a} \int_a^b g(x)dx
\]
\[
f_{avg}\cdot g_{avg} = \frac{1}{(b-a)^2}\int_a^b f(x)dx\int_a^b g(x)dx
\]

Thus: 
\[
(fg)_{avg} \neq f_{avg}\cdot g_{avg}
\]


\item This statement is false as:
\[
(fg)_{avg} = \frac{1}{b - a} \int_a^b f(x) g(x)dx
\]
\[
f_{avg} = \frac{1}{b - a} \int_a^b f(x)dx
\]
\[
g_{avg} = \frac{1}{b - a} \int_a^b g(x)dx
\]
\[
f_{avg}\cdot g_{avg} = \frac{b-a}{b-a}\frac{\int_a^b f(x)dx}{\int_a^b g(x)dx}
\]
\[
f_{avg}\cdot g_{avg} = \frac{\int_a^b f(x)dx}{\int_a^b g(x)dx}
\]

Thus:

\[
(fg)_{avg} \neq \frac{f_{avg}}{g_{avg}}
\]


\end{enumerate}

\item [35. ]
Assume for sake of contradiction that two distinct continuous functions $f(x)$ and $g(x)$ have the same average over every interval. Define $h(x) = f(x) - g(x)$. $|h(x)| > 0$ for at least 1 interval. 
Let $[a,b]$ represent any arbitrary interval. Thus, for any arbitrary interval, the averages are equal. 
\[
\frac{1}{b-a}\int_a^bf(x)dx = \frac{1}{b-a}\int_a^bg(x)dx = 0
\]
\[
\int_a^bf(x)dx - \int_a^bg(x)dx = 0
\]
\[
\int_a^b f(x) - g(x) dx = 0
\]
\[
\int_a^bh(x) dx= 0
\]
However, this is a contradiction as $h(x) > 0$ for at least one interval. The integral of $h(x)$ being 0 for all intervals implies that $h(x) = 0$ for all intervals, however, that contradicts our prior statement, thus disproving the assumption.

\item [40. ]
Here’s the full solution with the steps for evaluating the integral in LaTeX:

To find the area in the first quadrant bounded by the coordinate axes and the parabola \( y = 1 + a - ax^2 \) (where \( a > 0 \)), we proceed as follows:

1. Intersection Points:
   - Y-axis: At \( x = 0 \), \( y = 1 + a \).
   - X-axis: Set \( y = 0 \):
     \[
     1 + a - ax^2 = 0
     \]
     Solving for \( x \), we get:
     \[
     x^2 = \frac{1 + a}{a} \Rightarrow x = \sqrt{\frac{1 + a}{a}}
     \]
   So the parabola intersects the x-axis at \( x = \sqrt{\frac{1 + a}{a}} \).

2. Set up the Integral for the Area:  
   The area \( A \) in the first quadrant bounded by the parabola, the x-axis, and the y-axis is given by:
   \[
   A = \int_0^{\sqrt{\frac{1 + a}{a}}} \left(1 + a - ax^2\right) \, dx
   \]
   Expanding this integral:
   \[
   A = \int_0^{\sqrt{\frac{1 + a}{a}}} (1 + a - ax^2) \, dx
   \]
   We can separate the integral as follows:
   \[
   A = \int_0^{\sqrt{\frac{1 + a}{a}}} 1 \, dx + \int_0^{\sqrt{\frac{1 + a}{a}}} a \, dx - \int_0^{\sqrt{\frac{1 + a}{a}}} ax^2 \, dx
   \]
   Evaluating each term separately:
   
   - For the first term:
     \[
     \int_0^{\sqrt{\frac{1 + a}{a}}} 1 \, dx = \left[ x \right]_0^{\sqrt{\frac{1 + a}{a}}} = \sqrt{\frac{1 + a}{a}}
     \]
   
   - For the second term:
     \[
     \int_0^{\sqrt{\frac{1 + a}{a}}} a \, dx = a \cdot \left[ x \right]_0^{\sqrt{\frac{1 + a}{a}}} = a \cdot \sqrt{\frac{1 + a}{a}} = \sqrt{a(a + 1)}
     \]

   - For the third term:
     \[
     \int_0^{\sqrt{\frac{1 + a}{a}}} ax^2 \, dx = a \cdot \left[ \frac{x^3}{3} \right]_0^{\sqrt{\frac{1 + a}{a}}} = a \cdot \frac{\left( \sqrt{\frac{1 + a}{a}} \right)^3}{3} = \frac{a}{3} \cdot \frac{(1 + a)^{3/2}}{a^{3/2}} = \frac{2}{3} \sqrt{\frac{a + 1}{a}} (a + 1)
     \]

     Now that we have the function for area on basis of a, we will equate the derivative to 0 and obtain the value of that minimizes area.

     \[
     A'(a) =\frac{2}{3}\left(\sqrt{1 + \frac{1}{a}} - \frac{a + 1}{2 \sqrt{\frac{1}{a} + 1} \, a^{2}}
     \right)
     \]
     \[
     =\frac{2 \sqrt{\frac{1}{a} + 1}}{3} - \frac{a + 1}{3 \sqrt{\frac{1}{a} + 1} \, a^{2}}
     \]
     \[
     =\frac{2a^{2} + a - 1}{3 \sqrt{\frac{1}{a} + 1} \, a^{2}} = 0
     \]
     \[
     2a^2 + a -1 = 0
     \]
     \[
     (a+1)(a-0.5) = 0
     \]
     \[
     a=0.5 \quad or \quad a=-1
     \]
     We select $a=0.5$ as $a>0$ is given. 

  \item [10. ]
  Firstly, we find the intersections of the curves
  \[
  2-x = x^2
  \]
  \[
  x^2 -2 +x
  \]
  \[
  x=-2 \quad or \quad x=1
  \]
     \begin{enumerate}
       \item Around $y=0$:
       \[
       V = \pi\int_{-2}^1 (f(x))^2 - (g(x))^2dx
       \]
       \[
       V = \pi\int_{-2}^1 x^4 - 4 + 4x - x^2 dx
       \]
       \item Around $y=4$
       \[
       V = \pi\int_{-2}^1 (4-f(x))^2 - (4-g(x))^2dx
       \]
       \[
       V = \pi\int_{-2}^112 - 4 x - 9 x^2 + x^4dx
       \]
       \item Around $y = -1$
       \[
       V = \pi\int_{-2}^1 (-1-f(x))^2 - (-1-g(x))^2dx
       \]
       \[
       V = \pi\int_{-2}^1(-8 + 6 x + x^2 + x^4)dx
       \]
       \item Around $y=\frac{1}{4}$
       We first find intersections of $y=\frac{1}{4}$ with graph. 
       \[
       x^2 = \frac{1}{4}
       \]
       \[
       x = \pm 0.5
       \]
       So we setup the integral as:
       \[
       \pi\int_{-2}^{-0.5} \left(\frac{1}{4}-f(x)\right)^2 - \left(\frac{1}{4}-g(x)\right)^2 dx + \pi\int_{-0.5}^{0.5}\left(\frac{1}{4} - g(x)\right)^2dx
       \]
       \[
       +\pi\int_{0.5}^{2} \left(\frac{1}{4}-f(x)\right)^2 - \left(\frac{1}{4}-g(x)\right)^2 dx
       \]
       Plugging in functions:
       \[
       \pi\int_{-2}^{-0.5}0.5 (-6 + 7 x - 3 x^2 + 2 x^4)dx
       \]
       \[
       + \int_{-0.5}^{0.5}\frac{1}{16} - \frac{x^2}{2} + x^4 dx
       \]
       \[
       +\pi\int_{0.5}^{2}0.5 (-6 + 7 x - 3 x^2 + 2 x^4)dx
       \]
     \end{enumerate}

    \item [44. ]
    \begin{enumerate}

        \item Initial Setup:
        
        Given a hemispherical punch bowl with diameter 2 feet, the radius \( R \) is:
        \[
        R = \frac{2 \text{ feet}}{2} = 1 \text{ foot}
        \]
        
        Since the punch is filled to within 1 inch of the top, the initial height of punch from the bottom of the bowl is:
        \[
        H_{\text{initial}} = 1 - \frac{1}{12} = \frac{11}{12} \text{ feet}
        \]
        
        \item Initial Volume of Punch:
        
        We use the disk method to find the volume of punch, rotating the region around the \( y \)-axis from \( y = 0 \) to \( y = \frac{11}{12} \). The volume is:
        \[
        V_{\text{initial}} = \pi \int_0^{\frac{11}{12}} \left(\sqrt{1 - y^2}\right)^2 \, dy = \pi \int_0^{\frac{11}{12}} (1 - y^2) \, dy
        \]
        Calculating this integral:
        \[
        V_{\text{initial}} = \pi \int_0^{\frac{11}{12}} (1 - y^2) \, dy = \pi \left[ y - \frac{y^3}{3} \right]_0^{\frac{11}{12}} = \pi \left( \frac{11}{12} - \frac{\left(\frac{11}{12}\right)^3}{3} \right)
        \]
        Simplifying, we get:
        \[
        V_{\text{initial}} \approx 2.073 \text{ cubic feet}
        \]
        
        \item Final Volume of Punch:
        
        Thirty minutes later, there are only 2 inches of punch left, which is:
        \[
        H_{\text{final}} = \frac{1}{6} \text{ feet}
        \]
        
        Using the disk method, the volume is:
        \[
        V_{\text{final}} = \pi \int_0^{\frac{1}{6}} (1 - y^2) \, dy = \pi \left[ y - \frac{y^3}{3} \right]_0^{\frac{1}{6}} = \pi \left( \frac{1}{6} - \frac{\left(\frac{1}{6}\right)^3}{3} \right)
        \]
        Simplifying, we get:
        \[
        V_{\text{final}} \approx 0.519 \text{ cubic feet}
        \]
        
        \item Punch Consumed:
        
        The amount of punch consumed is the difference:
        \[
        V_{\text{consumed}} = V_{\text{initial}} - V_{\text{final}} \approx 2.073 - 0.519 = 1.554 \text{ cubic feet}
        \]
    \end{enumerate}
\end{enumerate}




    

\end{document}
