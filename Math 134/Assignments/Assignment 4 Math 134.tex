\documentclass[12pt]{article}

\usepackage{graphicx}			% Use this package to include images
\usepackage{amsmath}		
\usepackage{polynom}
% A library of many standard math expressions
\graphicspath{ {./Images/} }
\usepackage[margin=1in]{geometry}% Sets 1in margins. 
\usepackage{fancyhdr}			% Creates headers and footers
\usepackage{enumerate}          %These two package give custom labels to a list
\usepackage[shortlabels]{enumitem}


% Creates the header and footer. You can adjust the look and feel of these here.
\pagestyle{fancy}
\fancyhead[l]{Aditya Gupta}
\fancyhead[c]{Math 134 Homework \#4}
\fancyhead[r]{\today}
\fancyfoot[c]{\thepage}
\renewcommand{\headrulewidth}{0.2pt} %Creates a horizontal line underneath the header
\setlength{\headheight}{15pt} %Sets enough space for the header



\begin{document} 
\begin{enumerate}[start=1,label={\bfseries. },leftmargin=1in]

\item [29.]  We are given that \( f \) is differentiable on an interval \( I \) and that \( f'(x) < 1 \) for all \( x \in I \). To show that \( f \) has at most one fixed point in \( I \), we define the function

\[
g(x) = f(x) - x.
\]

A fixed point \( c \) is where \( f(c) = c \), which is equivalent to

\[
g(c) = 0.
\]

Since \( f \) is differentiable, \( g(x) \) is also differentiable, and its derivative is

\[
g'(x) = f'(x) - 1.
\]

Given that \( f'(x) < 1 \) for all \( x \in I \), it follows that

\[
g'(x) = f'(x) - 1 < 0,
\]

which means that \( g(x) \) is strictly decreasing on \( I \).

Now, assume for contradiction that \( f \) has two distinct fixed points, \( c_1 \) and \( c_2 \), where \( c_1 \neq c_2 \). This implies

\[
g(c_1) = g(c_2) = 0.
\]

By Rolle's Theorem, since \( g(x) \) is continuous and differentiable, there must be some point \( c_3 \in (c_1, c_2) \) such that

\[
g'(c_3) = 0.
\]

However, since \( g'(x) < 0 \) for all \( x \in I \), this contradicts Rolle's Theorem, implying that two distinct fixed points cannot exist. Thus, \( f \) can have at most one fixed point.

For zero fixed points, if \( g(a) > 0 \) and \( g(b) > 0 \) (or \( g(a) < 0 \) and \( g(b) < 0 \)), then \( g(x) \) does not cross the \( x \)-axis, meaning \( f(x) = x \) has no solution. Hence, there are no fixed points in \( I \).

For one fixed point, if \( g(a) \) and \( g(b) \) have different signs, the intermediate value theorem guarantees that there is some \( c \in (a, b) \) such that

\[
g(c) = 0,
\]

which means there is exactly one fixed point in \( I \).

\item [34. ]To show that the equation 
    \[
    x^n + ax + b = 0,
    \]
    where \( n \) is an odd positive integer, has at most three distinct real roots, we proceed as follows: 

    Let 
    \[
    f(x) = x^n + ax + b.
    \]

    The derivative of \( f(x) \) is given by:
    \[
    f'(x) = n x^{n-1} + a.
    \]

    To find critical points, we set the derivative equal to zero:
    \[
    f'(x) = 0 \implies n x^{n-1} + a = 0 \implies n x^{n-1} = -a \implies x^{n-1} = -\frac{a}{n}.
    \]

    - If \( a = 0 \): The derivative becomes 
    \[
    f'(x) = n x^{n-1},
    \]
    which has one critical point at \( x = 0 \).

    - If \( a > 0 \): The equation 
    \[
    x^{n-1} = -\frac{a}{n}
    \]
    has no real solutions because the left side cannot be negative and the right side is negative. Thus, there are no critical points.

    - If \( a < 0 \): The equation 
    \[
    x^{n-1} = -\frac{a}{n}
    \]
    has two distinct real solutions because \( n-1 \) is even, allowing for two critical points.

    Now, assume for contradiction that \( f(x) = 0 \) has four distinct real roots, say \( r_1, r_2, r_3, r_4 \).

    By Rolle's Theorem, since \( f(x) \) is continuous and differentiable, there must be at least one critical point between \( r_1 \) and \( r_2 \), and another between \( r_2 \) and \( r_3 \) and another between $r_3$ and $r_4$. This implies that \( f(x) \) has at least three critical points, which contradicts the earlier finding that \( f(x) \) can have at most two critical points.

    Therefore, we conclude that the equation 
    \[
    x^n + ax + b = 0
    \]
    can have at most three distinct real roots when \( n \) is an odd positive integer.

\item [59. ]
\begin{enumerate}

    \item Let \( f(x) = (1 + x)^n - (1 + nx) \), where \( n > 1 \) and \( x > 0 \). We aim to prove that \( f(x) > 0 \) for all \( x > 0 \). 
    
    First, evaluate \( f(x) \) at \( x = 0 \):
    \[
    f(0) = (1 + 0)^n - (1 + n \cdot 0) = 1 - 1 = 0
    \]
    Hence, \( f(0) = 0 \).

    Next, compute the derivative of \( f(x) \):
    \[
    f'(x) = \frac{d}{dx} \left( (1 + x)^n - (1 + nx) \right)
    \]
    Using the chain rule, we get
    \[
    f'(x) = n(1 + x)^{n-1} - n
    \]
    Factoring out \( n \), we have
    \[
    f'(x) = n \left( (1 + x)^{n-1} - 1 \right)
    \]
    Since \( n > 1 \) and \( x > 0 \), it follows that \( (1 + x)^{n-1} > 1 \), which implies \( f'(x) > 0 \) for all \( x > 0 \).

    Now, assume for contradiction that $f(c_1) = 0$ for $c_1 > 0$. By Rolle's Theorem, there must be a point \( c \in (0, c_1) \) where \( f'(c) = 0\).

    For \( f'(c) = 0 \), we require
    \[
    (1 + c)^{n-1} = 1
    \]
    which implies \( 1 + c = 1 \), or \( c = 0 \). This contradicts the assumption that $c>0$

    Therefore, the assumption that \( f(c_1) = 0 \) must be false, and we conclude that \( f(x) > 0 \) for all \( x > 0 \). Thus, \( (1 + x)^n > 1 + nx \) for all \( x > 0 \) and \( n > 1 \).

    \item We aim to prove by induction that \((1 + x)^n > 1 + nx\) for all \( x > 0 \) and \( n \geq 2 \).
    
    \textbf{Base Case:} For \( n = 2 \), we want to show:
    \[
    (1 + x)^2 > 1 + 2x
    \]
    Expanding both sides:
    \[
    (1 + x)^2 = 1 + 2x + x^2
    \]
    Since \( x^2 > 0 \) for all \( x > 0 \), it follows that:
    \[
    1 + 2x + x^2 > 1 + 2x
    \]
    Thus, the inequality holds for \( n = 2 \).

    \textbf{Inductive Hypothesis:} Assume the inequality holds for some \( n = k \), i.e., 
    \[
    (1 + x)^k > 1 + kx
    \]
    This implies:
    \[
    (1 + x)^k - (1 + kx) > 0 \quad \text{for all} \quad x > 0.
    \]

    \textbf{Inductive Step:} We now need to prove the inequality for \( n = k+1 \), i.e., show that:
    \[
    (1 + x)^{k+1} > 1 + (k+1)x
    \]
    Take everything to the left-hand side (LHS):
    \[
    (1 + x)^{k+1} - (1 + (k+1)x) > 0
    \]
    Express \( (1 + x)^{k+1} \) as \( (1 + x)^k \cdot (1 + x) \):
    \[
    (1 + x)^k \cdot (1 + x) - (1 + (k+1)x)
    \]
    Expanding both sides, we get:
    \[
    (1 + x)^k \cdot (1 + x) = (1 + x)^k + x(1 + x)^k
    \]
    and
    \[
    1 + (k+1)x = 1 + kx + x.
    \]
    Substituting these into the expression gives:
    \[
    \left( (1 + x)^k + x(1 + x)^k \right) - (1 + kx + x)
    \]
    Simplifying further, we have:
    \[
    (1 + x)^k - (1 + kx) + x \left( (1 + x)^k - 1 \right)
    \]
    By the inductive hypothesis, \( (1 + x)^k - (1 + kx) > 0 \) for all \( x > 0 \). Moreover, since \( (1 + x)^k > 1 \) for \( x > 0 \) and \( k \geq 1 \), the term \( x \left( (1 + x)^k - 1 \right) > 0 \) for all \( x > 0 \).

    Since each term is positive, we conclude that:
    \[
    (1 + x)^{k+1} - (1 + (k+1)x) > 0 \quad \text{for all} \quad x > 0.
    \]
\end{enumerate}

\item [42.]

To begin:
\[
D = \sqrt{x^2 + f(x)^2}
\]
Differentiating with respect to x using chain rule,
\[
D' = \frac{2f(x)f'(x) + 2x}{2 \sqrt{f(x)^{2} + x^{2}}}
\]

Since D has an extrema at c, $D'(c) = 0$
\[
\frac{2f(c)f'(c) + 2c}{2 \sqrt{f(c)^{2} + c^{2}}} = 0
\]
\[
2f(c)f'(c) + 2c = 0
\]
\[
f(c)f'(c) = -c
\]
\[
f'(c) = \frac{-c}{f(c)}
\]

Thus at point c, the slope of the tangent to $f(x)$ at c is $ m_1 = \frac{-c}{f(c)}$ 

The line passing through $(0,0)$ and $(c,f(c)$ has a slope of:
\[
m_2 = \frac{f(c) - 0}{c-0} = \frac{f(c)}{c}
\]

Now, because $m_1 \times m_2 = -1$, these lines are perpendicular to each other.

\item [39.]
If \( f \) is continuous on \([a, b]\) and \( f(a) = f(b) \), we want to show that \( f \) has at least one critical point in \((a, b)\).
\begin{enumerate}
    \item If \( f \) is constant on \([a, b]\), then \( f'(x) = 0 \) for all \( x \in (a, b) \), meaning every point in \((a, b)\) is a critical point.
    \item If \( f \) is not constant and differentiable, we can apply the Mean Value Theorem. According to the Mean Value Theorem, there exists some point \( c \in (a, b) \) such that:
   \[
   f'(c) = \frac{f(b) - f(a)}{b - a} = \frac{0}{b - a} = 0.
   \]
   Thus, \( c \) is a critical point.
   \item If \( f \) is not differentiable at some point \( c \in (a, b) \), then \( c \) is also a critical point because critical points include points where the derivative does not exist.
\end{enumerate}


\item [41. ] An example of a function that satisfies the conditions is the Dirichlet function:

f(x) = \begin{cases}
    1, & x\in \mathbb{Q}\\
    0, & x\notin \mathbb{Q}
\end{cases}

This satisfies all conditions.

\item [48. ]
\begin{enumerate}
    \item 
   The first derivative is:
   \[
   p'(x) = 3x^2 + 2ax + b
   \]

   Setting \( p'(x) = 0 \):
   \[
   3x^2 + 2ax + b = 0
   \]

   Using the quadratic formula:
   \[
   x_{1,2} = \frac{-2a \pm \sqrt{(2a)^2 - 4 \cdot 3 \cdot b}}{2 \cdot 3} = \frac{-2a \pm \sqrt{4a^2 - 12b}}{6} = \frac{-a \pm \sqrt{a^2 - 3b}}{3}
   \]

   Let \( x_1 = \frac{-a + \sqrt{a^2 - 3b}}{3} \)  and \( x_2 = \frac{-a - \sqrt{a^2 - 3b}}{3} \).


    \item Calculate the midpoint.

   The midpoint \( M \) of \( x_1 \) and \( x_2 \):
   \[
   M = \frac{x_1 + x_2}{2} = \frac{\left(\frac{-a + \sqrt{a^2 - 3b}}{3}\right) + \left(\frac{-a - \sqrt{a^2 - 3b}}{3}\right)}{2} = \frac{-2a}{6} = \frac{-a}{3}
   \]

    \item Show that \( M \) is a point of inflection.

   The second derivative is:
   \[
   p''(x) = 6x + 2a
   \]

   Setting \( p''(x) = 0 \):
   \[
   6x + 2a = 0 \quad \Rightarrow \quad x = -\frac{a}{3}
   \]

   Thus, the midpoint \( M = -\frac{a}{3} \) is where \( p''(x) = 0 \), indicating a point of inflection.

    \item 
    For sake of brevity and ease of calculation, we can express \( p'(x) = e(x - r)(x - s) \), where \( r \) and \( s \) are the critical points and roots of the first derivative. Expanding this:

    \[
    p'(x) = e(x^2 - (r+s)x + rs)
    \]

    To find \( p(x) \), we can reverse differentiate $p'(x)$:

    \[
    p(x) = e\left(\frac{x^3}{3} - \frac{(r+s)x^2}{2} + rsx\right) + C
    \]

    Now, we calculte \( p(r) \) and \( p(s) \), the \( y \)-coordinates at the critical points:

    \[
    p(r) = e\left(\frac{r^3}{3} - \frac{(r+s)r^2}{2} + rsr\right) + C
    \]

    \[
    p(s) = e\left(\frac{s^3}{3} - \frac{(r+s)s^2}{2} + rss\right) + C
    \]

    Next, we compute the averages of \( p(r) \) and \( p(s) \):

    \[
    \frac{p(r) + p(s)}{2} = \frac{e\left[\left(\frac{r^3}{3} - \frac{(r+s)r^2}{2} + r^2s\right) + \left(\frac{s^3}{3} - \frac{(r+s)s^2}{2} + rs^2\right)\right] + 2C}{2}
    \]

    Simplifing:

    \[
    = \frac{e\left[\frac{r^3 + s^3}{3} - \frac{(r+s)(r^2 + s^2)}{2} + rs(r+s)\right] + 2C}{2}
    \]

    Using the following expression: \( r^3 + s^3 = (r+s)(r^2 - rs + s^2) \), this becomes:

    \[
    = e\left(\frac{(r+s)((r+s)^2 - 3rs)}{12}\right) + C
    \]

    Now, compute \( p\left( \frac{r+s}{2} \right) \), the \( y \)-coordinate at the midpoint \( M_x = \frac{r+s}{2} \):

    \[
    p(M_x) = e\left[\frac{\left(\frac{r+s}{2}\right)^3}{3} - \frac{(r+s)\left(\frac{r+s}{2}\right)^2}{2} + rs\left(\frac{r+s}{2}\right)\right] + C
    \]

    Simplifying:

    \[
    = e\left[\frac{(r+s)^3}{24} - \frac{(r+s)^3}{8} + \frac{rs(r+s)}{2}\right] + C
    \]

    Factoring out common terms:

    \[
    p(M_x) = e\left[\frac{(r+s)((r+s)^2 - 3rs)}{12}\right] + C
    \]

    This shows that the \( y \)-coordinate at the midpoint is the same as the average of \( p(r) \) and \( p(s) \), confirming that the midpoint lies at the point of inflection.



\end{enumerate}

\item [52. ]

\[
f(x) = \frac{1+x-3x^2}{x}
\]

The vertical asymptote is $x=0$

The oblique asymptote is found by division:


\polylongdiv{1+x-3x^2}{x}
\newpage
Thus, the oblique asymptote is $y=-3x+1$

\includegraphics[width=1\linewidth]{Math 134//Images/Screenshot 2024-10-20 at 7.39.22 PM.png}

\item [48. ]
\begin{enumerate}
    \item 
    \[
    h = \frac{gt^2}{2}
    \]
    \[
    h = \frac{9 \cdot 9.8}{2} = 44.1m = 144.69 \text{ ft}
    \]

    \item 
    Speed of sound = 1080 ft/s

    Time spent by sound to travel up from water = $\frac{h}{1080}$

    \[
    h = \frac{1}{2} \cdot g \cdot \left(3-\frac{h}{1800}\right)^2
    \]
    \[
    h = 16.1\left(9 - \frac{6h}{1800} + \frac{h^2}{3,240,000}\right)
    \]
    Factorising of this function is not feasible, thus, I obtain the value of h:

    \[
    h =133.038 ft
    \]
\end{enumerate}

\item [20.]
\[
dv = V'(r) \cdot h
\]
\[
V'(r) = 4\pi r^2
\]
\[
\frac{8 \times 10^6}{4 \pi \times 4000^2} = h
\]
\[
h = 0.00397 \text{ miles} \approx 64 m
\]

\item Worksheet 4:
\begin{enumerate}
    \item
    The cone has radius \(b\) and height \(a\). Let the cylinder's radius be \(r\) and height \(h\). Using similar triangles, the radius of the cone at height \(h\) is \(\frac{b(a-h)}{a}\). Hence, \(r = \frac{b(a-h)}{a}\) for the cylinder to fit inside.
    
    \item  
    The volume \(V\) of the cylinder is:
    \[
    V = \pi r^2 h = \pi \left(\frac{b(a-h)}{a}\right)^2 h = \pi \frac{b^2 (a-h)^2 h}{a^2}.
    \]

    \item  
    Differentiating \(V\):
    \[
    \frac{dV}{dh} = \pi \frac{b^2}{a^2} \left[ 2(a-h)(-1) h + (a-h)^2 \right] = \pi \frac{b^2}{a^2} (a-h)(a - 3h).
    \]
    Equating \(\frac{dV}{dh} = 0\) to find the critical points:  
    \[
    (a-h)(a - 3h) = 0.
    \]     This gives two solutions: \(h = a\) and \(h = \frac{a}{3}\). Since \(h = a\)) does not work, \(h = \frac{a}{3}\) is the selected h size

    \item 
    Substituting \(h = \frac{a}{3}\) into the expression for \(r\):
    \[
    r = \frac{b(a - \frac{a}{3})}{a} = \frac{2b}{3}.
    \]
    The maximum volume is:
    \[
    V = \pi \left(\frac{2b}{3}\right)^2 \cdot \frac{a}{3} = \frac{4\pi b^2 a}{27}.
    \]
\end{enumerate}

\end{enumerate}
\end{document}
