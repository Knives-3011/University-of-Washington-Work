\documentclass[12pt]{article}
\usepackage{bigints}
\usepackage{graphicx}			% Use this package to include images
\usepackage{amsmath}	
\usepackage{amssymb}
\usepackage{amsfonts}
\usepackage{polynom}
\usepackage{listings}
% A library of many standard math expressions
\graphicspath{ {./Images/} }
\usepackage[margin=1in]{geometry}% Sets 1in margins. 
\newcommand{\qed}[0]{$\blacksquare$}
\usepackage{fancyhdr}			% Creates headers and footers
\usepackage{enumerate}          %These two package give custom labels to a list
\usepackage[shortlabels]{enumitem}


% Creates the header and footer. You can adjust the look and feel of these here.
\pagestyle{fancy}
\fancyhead[l]{Aditya Gupta}
\fancyhead[c]{Math 135 Homework \#10}
\fancyhead[r]{\today}
\fancyfoot[c]{\thepage}
\renewcommand{\headrulewidth}{0.2pt} %Creates a horizontal line underneath the header
\setlength{\headheight}{15pt} %Sets enough space for the header
\begin{document}
\begin{enumerate}
\item 
\begin{enumerate}
    \item

    Since \( \mathbf{G}(t) \) is an antiderivative of \( \mathbf{f}(t) \), we have:
    \[
    \mathbf{G}'(t) = \mathbf{f}(t).
    \]
    Substituting this into the integral, we get:
    \[
    \int_a^b \mathbf{f}(t) dt = \int_a^b \mathbf{G}'(t) dt.
    \]
    By the Fundamental Theorem of Calculus:
    \[
    \int_a^b \mathbf{G}'(t) dt = \mathbf{G}(b) - \mathbf{G}(a).
    \]
    Thus, the result follows:
    \[
    \int_a^b \mathbf{f}(t) dt = \mathbf{G}(b) - \mathbf{G}(a).
    \]

    \item 

    Let \( \mathbf{F} \) and \( \mathbf{G} \) be two antiderivatives of \( \mathbf{f} \), meaning:
    \[
    \mathbf{F}'(t) = \mathbf{f}(t), \quad \mathbf{G}'(t) = \mathbf{f}(t) \quad \text{for all } t \in I.
    \]
    Define \( \mathbf{H}(t) = \mathbf{F}(t) - \mathbf{G}(t) \). Then,
    \[
    \mathbf{H}'(t) = \mathbf{F}'(t) - \mathbf{G}'(t) = \mathbf{f}(t) - \mathbf{f}(t) = \mathbf{0}.
    \]
    Since \( \mathbf{H}'(t) = \mathbf{0} \) for all \( t \in I \), \( \mathbf{H}(t) \) must be a constant vector, say \( \mathbf{C} \). That is,
    \[
    \mathbf{F}(t) - \mathbf{G}(t) = \mathbf{C}.
    \]
    Rearranging gives:
    \[
    \mathbf{F} = \mathbf{G} + \mathbf{C},
    \]
    where \( \mathbf{C} \) is a constant vector. \qed
\end{enumerate}
\item 
\[
\mathbf{g}(t) = \frac{\mathbf{f}(t)}{\|\mathbf{f}(t)\|}
\]

We need to compute:

\[
\frac{d}{dt} \left( \frac{\mathbf{f}(t)}{\|\mathbf{f}(t)\|} \right)
\]

Using the quotient rule:

\[
\frac{d}{dt} \left( \frac{\mathbf{f}(t)}{\|\mathbf{f}(t)\|} \right) =
\frac{\mathbf{f}'(t) \|\mathbf{f}(t)\| - \mathbf{f}(t) \frac{d}{dt} \|\mathbf{f}(t)\|}{\|\mathbf{f}(t)\|^2}
\]

Since 

\[
\|\mathbf{f}(t)\| = \sqrt{\mathbf{f}(t) \cdot \mathbf{f}(t)}
\]

differentiate using the chain rule:

\[
\frac{d}{dt} \|\mathbf{f}(t)\| = \frac{1}{2} (2 \mathbf{f}(t) \cdot \mathbf{f}'(t)) \cdot \|\mathbf{f}(t)\|^{-1}
= \frac{\mathbf{f}(t) \cdot \mathbf{f}'(t)}{\|\mathbf{f}(t)\|}
\]

Substituting this:

\[
\frac{d}{dt} \left( \frac{\mathbf{f}(t)}{\|\mathbf{f}(t)\|} \right) =
\frac{\mathbf{f}'(t) \|\mathbf{f}(t)\| - \mathbf{f}(t) \frac{\mathbf{f}(t) \cdot \mathbf{f}'(t)}{\|\mathbf{f}(t)\|}}{\|\mathbf{f}(t)\|^2}
\]

Factor out \(\|\mathbf{f}(t)\|\):

\[
= \frac{\mathbf{f}'(t) \|\mathbf{f}(t)\|}{\|\mathbf{f}(t)\|^2} - \frac{\mathbf{f}(t) (\mathbf{f}(t) \cdot \mathbf{f}'(t))}{\|\mathbf{f}(t)\|^3}
\]

\[
= \frac{\mathbf{f}'(t)}{\|\mathbf{f}(t)\|} - \frac{\mathbf{f}(t) \cdot \mathbf{f}'(t)}{\|\mathbf{f}(t)\|^3} \mathbf{f}(t)
\]

Thus, the required result is:

\[
\frac{d}{dt} \left( \frac{\mathbf{f}(t)}{\|\mathbf{f}(t)\|} \right) =
\frac{\mathbf{f}'(t)}{\|\mathbf{f}(t)\|} - \frac{\mathbf{f}(t) \cdot \mathbf{f}'(t)}{\|\mathbf{f}(t)\|^3} \mathbf{f}(t).
\]

\item 
The curvature is defined as:

\[
\kappa = \left\| \frac{d\mathbf{T}}{ds} \right\|
\]

where the unit tangent vector is given by:

\[
\mathbf{T} = \frac{\mathbf{r'}}{\|\mathbf{r'}\|}
\]

Using the chain rule,

\[
\frac{d\mathbf{T}}{ds} = \frac{d\mathbf{T}}{dt} \cdot \frac{dt}{ds}
\]

From the Fundamental Theorem of Calculus,

\[
\frac{ds}{dt} = \|\mathbf{r'}(t)\|
\]

which implies

\[
\frac{dt}{ds} = \frac{1}{\|\mathbf{r'}(t)\|}
\]

Substituting this, we obtain

\[
\kappa = \left\| \frac{d\mathbf{T}}{dt} \right\| \frac{1}{\|\mathbf{r'}\|}
\]

or equivalently,

\[
\kappa = \frac{\| d\mathbf{T}/dt \|}{\|\mathbf{r'}\|}
\]

Since 

\[
\mathbf{T} = \frac{\mathbf{r'}}{\|\mathbf{r'}\|}
\]

differentiating both sides using the product rule gives:

\[
\frac{d\mathbf{T}}{dt} = \frac{1}{\|\mathbf{r'}\|} \mathbf{r''} - \frac{(\mathbf{r'} \cdot \mathbf{r''}) \mathbf{r'}}{\|\mathbf{r'}\|^3}
\]

Taking the magnitude and using the cross product identity:

\[
\left\| \frac{d\mathbf{T}}{dt} \right\| = \frac{\|\mathbf{r'} \times \mathbf{r''} \|}{\|\mathbf{r'}\|^2}
\]

Substituting this into the curvature formula,

\[
\kappa = \frac{\| \mathbf{r'} \times \mathbf{r''} \|}{\|\mathbf{r'}\|^3}
\]

Thus, the formula is proven. \qed

\end{enumerate}
\end{document}
