\documentclass[12pt]{article}
\usepackage{bigints}
\usepackage{graphicx}			% Use this package to include images
\usepackage{amsmath}	
\usepackage{amssymb}
\usepackage{amsfonts}
\usepackage{polynom}
\usepackage{listings}
% A library of many standard math expressions
\graphicspath{ {./Images/} }
\usepackage[margin=1in]{geometry}% Sets 1in margins. 
\newcommand{\qed}[0]{$\blacksquare$}
\usepackage{fancyhdr}			% Creates headers and footers
\usepackage{enumerate}          %These two package give custom labels to a list
\usepackage[shortlabels]{enumitem}
% Creates the header and footer. You can adjust the look and feel of these here.
\pagestyle{fancy}
\fancyhead[l]{Aditya Gupta}
\fancyhead[c]{Math 135 Homework \#9}
\fancyhead[r]{\today}
\fancyfoot[c]{\thepage}
\renewcommand{\headrulewidth}{0.2pt} %Creates a horizontal line underneath the header
\setlength{\headheight}{15pt} %Sets enough space for the header
\begin{document}
\begin{enumerate}
\item We begin with the standard triangle inequality:

\[
\|\mathbf{a}\| = \|\mathbf{a} - \mathbf{b} + \mathbf{b}\| \leq \|\mathbf{a} - \mathbf{b}\| + \|\mathbf{b}\|
\]

Rearranging this gives:

\[
\|\mathbf{a}\| - \|\mathbf{b}\| \leq \|\mathbf{a} - \mathbf{b}\|
\]

Similarly, applying the same property to \(\mathbf{b} - \mathbf{a}\), we obtain:

\[
\|\mathbf{b}\| = \|\mathbf{b} - \mathbf{a} + \mathbf{a}\| \leq \|\mathbf{b} - \mathbf{a}\| + \|\mathbf{a}\| = \|\mathbf{a} - \mathbf{b}\| + \|\mathbf{a}\|
\]

Rearranging this yields:

\[
\|\mathbf{b}\| - \|\mathbf{a}\| \leq \|\mathbf{a} - \mathbf{b}\|
\]

Since both inequalities hold, we conclude:

\[
|\|\mathbf{a}\| - \|\mathbf{b}\|| \leq \|\mathbf{a} - \mathbf{b}\|
\]

Thus, we see that:

\[
\|\mathbf{a} - \mathbf{b}\| \geq |\|\mathbf{a}\| - \|\mathbf{b}\||
\]

as required.


\item 
To prove the parallelogram law, we start with the given equation:

\[
\|\mathbf{a} + \mathbf{b}\|^2 + \|\mathbf{a} - \mathbf{b}\|^2 = 2\|\mathbf{a}\|^2 + 2\|\mathbf{b}\|^2.
\]

Expanding the norms using the dot product,

\[
\|\mathbf{a} + \mathbf{b}\|^2 = (\mathbf{a} + \mathbf{b}) \cdot (\mathbf{a} + \mathbf{b}).
\]

Expanding the right-hand side,

\[
\|\mathbf{a} + \mathbf{b}\|^2 = \mathbf{a} \cdot \mathbf{a} + 2 \mathbf{a} \cdot \mathbf{b} + \mathbf{b} \cdot \mathbf{b}.
\]

Similarly, for \(\|\mathbf{a} - \mathbf{b}\|^2\),

\[
\|\mathbf{a} - \mathbf{b}\|^2 = (\mathbf{a} - \mathbf{b}) \cdot (\mathbf{a} - \mathbf{b}).
\]

Expanding this expression,

\[
\|\mathbf{a} - \mathbf{b}\|^2 = \mathbf{a} \cdot \mathbf{a} - 2 \mathbf{a} \cdot \mathbf{b} + \mathbf{b} \cdot \mathbf{b}.
\]

Adding the two equations,

\[
\|\mathbf{a} + \mathbf{b}\|^2 + \|\mathbf{a} - \mathbf{b}\|^2 =
(\mathbf{a} \cdot \mathbf{a} + 2 \mathbf{a} \cdot \mathbf{b} + \mathbf{b} \cdot \mathbf{b}) +
(\mathbf{a} \cdot \mathbf{a} - 2 \mathbf{a} \cdot \mathbf{b} + \mathbf{b} \cdot \mathbf{b}).
\]

The terms \(+2 \mathbf{a} \cdot \mathbf{b}\) and \(-2 \mathbf{a} \cdot \mathbf{b}\) cancel out, leaving

\[
\|\mathbf{a} + \mathbf{b}\|^2 + \|\mathbf{a} - \mathbf{b}\|^2 = 2 \mathbf{a} \cdot \mathbf{a} + 2 \mathbf{b} \cdot \mathbf{b}.
\]

Since \(\mathbf{a} \cdot \mathbf{a} = \|\mathbf{a}\|^2\) and \(\mathbf{b} \cdot \mathbf{b} = \|\mathbf{b}\|^2\), we get

\[
\|\mathbf{a} + \mathbf{b}\|^2 + \|\mathbf{a} - \mathbf{b}\|^2 = 2 \|\mathbf{a}\|^2 + 2 \|\mathbf{b}\|^2.
\]

Thus, the parallelogram law is proved.
\item 
Let the sphere be centered at the origin in a 3D Cartesian coordinate system. If \( P_1 \) and \( P_2 \) are antipodal points, we can represent them as:
\[
P_1 = (x, y, z), \quad P_2 = (-x, -y, -z)
\]
where both lie on the sphere. Let \( Q \) be any other point on the sphere:
\[
Q = (a, b, c).
\]
The vectors from \( P_1 \) and \( P_2 \) to \( Q \) are:
\[
\overrightarrow{P_1Q} = (a-x, b-y, c-z)
\]
\[
\overrightarrow{P_2Q} = (a+x, b+y, c+z).
\]
To check if these vectors are perpendicular, we compute their dot product:
\[
\overrightarrow{P_1Q} \cdot \overrightarrow{P_2Q} = (a-x, b-y, c-z) \cdot (a+x, b+y, c+z).
\]
Expanding each term:
\[
(a-x)(a+x) + (b-y)(b+y) + (c-z)(c+z).
\]
Using the identity \( (m-n)(m+n) = m^2 - n^2 \), we get:
\[
a^2 - x^2 + b^2 - y^2 + c^2 - z^2.
\]
Since \( P_1 \) and \( P_2 \) are on the sphere of radius \( R \), we know:
\[
x^2 + y^2 + z^2 = R^2, \quad a^2 + b^2 + c^2 = R^2.
\]
Substituting these values:
\[
a^2 + b^2 + c^2 - (x^2 + y^2 + z^2) = R^2 - R^2 = 0.
\]
Since the dot product is zero, it follows that:
\[
\overrightarrow{P_1Q} \perp \overrightarrow{P_2Q}.
\]
Thus, the vectors are perpendicular, proving the required result.

\item If \(\mathbf{a}, \mathbf{b}, \mathbf{c}\) are linearly independent, then any vector \(\mathbf{d}\) can be written as a unique linear combination:

\[
\mathbf{d} = \alpha \mathbf{a} + \beta \mathbf{b} + \gamma \mathbf{c}
\]

Taking the dot product of both sides with \(\mathbf{b} \times \mathbf{c}\):

\[
\mathbf{d} \cdot (\mathbf{b} \times \mathbf{c}) = (\alpha \mathbf{a} + \beta \mathbf{b} + \gamma \mathbf{c}) \cdot (\mathbf{b} \times \mathbf{c})
\]

Using the scalar triple product properties, we get:

\[
\alpha (\mathbf{a} \cdot (\mathbf{b} \times \mathbf{c})) + \beta (\mathbf{b} \cdot (\mathbf{b} \times \mathbf{c})) + \gamma (\mathbf{c} \cdot (\mathbf{b} \times \mathbf{c})) = \mathbf{d} \cdot (\mathbf{b} \times \mathbf{c})
\]

Since \(\mathbf{b} \cdot (\mathbf{b} \times \mathbf{c}) = 0\) and \(\mathbf{c} \cdot (\mathbf{b} \times \mathbf{c}) = 0\), it simplifies to:

\[
\alpha (\mathbf{a} \cdot (\mathbf{b} \times \mathbf{c})) = \mathbf{d} \cdot (\mathbf{b} \times \mathbf{c})
\]

Solving for \(\alpha\):

\[
\alpha = \frac{\mathbf{d} \cdot (\mathbf{b} \times \mathbf{c})}{\mathbf{a} \cdot (\mathbf{b} \times \mathbf{c})}
\]

Similarly, applying the same method for \(\beta\) and \(\gamma\):

\[
\beta = \frac{\mathbf{d} \cdot (\mathbf{c} \times \mathbf{a})}{\mathbf{b} \cdot (\mathbf{c} \times \mathbf{a})}, \quad
\gamma = \frac{\mathbf{d} \cdot (\mathbf{a} \times \mathbf{b})}{\mathbf{c} \cdot (\mathbf{a} \times \mathbf{b})}
\]

Thus, the coefficients \(\alpha, \beta, \gamma\) are uniquely determined using the scalar triple products.

\item 
Let the semicircle be centered at the origin \( \mathbf{O} = (0,0) \) with radius \( a \). The endpoints of the diameter are
\( \mathbf{A} = (-a,0) \) and \( \mathbf{B} = (a,0) \).

Let \( \mathbf{C} = (x,y) \) be any point on the semicircle. Since it lies on the semicircle, it satisfies the equation:

\[
x^2 + y^2 = a^2
\]

Now, define the vectors \( \mathbf{AB}, \mathbf{AC}, \) and \( \mathbf{BC} \):

\[
\mathbf{AB} = \mathbf{B} - \mathbf{A} = (a,0) - (-a,0) = (2a,0)
\]

\[
\mathbf{AC} = \mathbf{C} - \mathbf{A} = (x - (-a), y - 0) = (x + a, y)
\]

\[
\mathbf{BC} = \mathbf{C} - \mathbf{B} = (x - a, y - 0) = (x - a, y)
\]

To prove that \( \angle ACB = 90^\circ \), we compute the dot product \( \mathbf{AC} \cdot \mathbf{BC} \):

\[
\mathbf{AC} \cdot \mathbf{BC} = (x + a, y) \cdot (x - a, y)
\]

Expanding using the dot product formula:

\[
(x + a)(x - a) + y \cdot y = x^2 - a^2 + y^2
\]

Since \( C(x,y) \) lies on the semicircle, substituting \( x^2 + y^2 = a^2 \) gives:

\[
(x^2 - a^2) + y^2 = a^2 - a^2 = 0
\]

Since the dot product is zero, \( \mathbf{AC} \) and \( \mathbf{BC} \) are perpendicular, proving that \( \angle ACB \) is a right angle.

\end{enumerate}
\end{document}
