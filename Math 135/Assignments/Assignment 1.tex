\documentclass[12pt]{article}
\usepackage{bigints}
\usepackage{graphicx}			% Use this package to include images
\usepackage{amsmath}	
\usepackage{amssymb}
\usepackage{amsfonts}
\usepackage{polynom}
% A library of many standard math expressions
\graphicspath{ {./Images/} }
\usepackage[margin=1in]{geometry}% Sets 1in margins. 
\newcommand{\qed}[0]{$\blacksquare$}
\usepackage{fancyhdr}			% Creates headers and footers
\usepackage{enumerate}          %These two package give custom labels to a list
\usepackage[shortlabels]{enumitem}


% Creates the header and footer. You can adjust the look and feel of these here.
\pagestyle{fancy}
\fancyhead[l]{Aditya Gupta}
\fancyhead[c]{Math 135 Homework \#1}
\fancyhead[r]{\today}
\fancyfoot[c]{\thepage}
\renewcommand{\headrulewidth}{0.2pt} %Creates a horizontal line underneath the header
\setlength{\headheight}{15pt} %Sets enough space for the header
\begin{document}


\begin{enumerate}
\item 
We are given that \(b_n\) is bounded, so \(|b_n| \leq M\) for all \(n\), where \(M\) is some real number. We also know that \(a_n \to 0\), so the limit of \(a_n\) is \(0\).


For every \(\epsilon > 0\), there exists a positive integer \(K\) such that if \(n \geq K\), then:
\[
|a_n| < \frac{\epsilon}{M}.
\]

Now, consider the product \(a_n b_n\). We know:
\[
|a_n b_n| = |a_n| \cdot |b_n|.
\]

Since \(|b_n| \leq M\), we have:
\[
|a_n b_n| \leq |a_n| \cdot M.
\]

For \(n \geq K\), substituting \(|a_n| < \frac{\epsilon}{M}\), we get:
\[
|a_n b_n| \leq \frac{\epsilon}{M} \cdot M = \epsilon.
\]

Thus, for all \(n \geq K\), \(|a_n b_n| < \epsilon\), implying $a_n \cdot b_n \to 0$.

\item First, we prove the sequence is bounded below by \(\sqrt{2}\). The recurrence relation is given as
\[
a_{n+1} = \frac{a_n^2 + 2}{2a_n}
\]
with the initial value \(a_1 = \frac{3}{2}\). Clearly, \(a_1 > \sqrt{2}\). Assume \(a_n \geq \sqrt{2}\). To show \(a_{n+1} \geq \sqrt{2}\), we use the recurrence relation:
\[
a_{n+1} = \frac{a_n^2 + 2}{2a_n}.
\]
Taking the following expression:
\[
(a_n - \sqrt{2})^2 \geq 0
\]
The above is true as squares are always non-negative.
Expanding out,

\[
a^2_{n} -2\sqrt{2}a_n + 2 \geq 0
\]

\[
a^2_{n} + 2 \geq 2\sqrt{2}a_n
\]

Dividing by $2a_n$

\[
\frac{a^2_{n} + 2}{2a_n} \geq \sqrt{2}
\]

Replacing in the recurrence relation:

\[
a_{n+1} \geq \sqrt{2}
\]
Thus, as long as $a_n \geq \sqrt{2}$
Next, we prove the sequence is decreasing. To show \(a_{n+1} \leq a_n\), consider the difference:
\[
a_{n+1} - a_n = \frac{a_n^2 + 2}{2a_n} - a_n.
\]
Simplify the expression:
\[
a_{n+1} - a_n = \frac{a_n^2 + 2 - 2a_n^2}{2a_n} = \frac{2 - a_n^2}{2a_n}.
\]
For \(a_n \geq \sqrt{2}\), we have \(a_n^2 \geq 2\), so
\[
2 - a_n^2 \leq 0.
\]
Since \(2a_n > 0\), it follows that
\[
\frac{2 - a_n^2}{2a_n} \leq 0.
\]
Hence,
\[
a_{n+1} - a_n \leq 0,
\]
which implies \(a_{n+1} \leq a_n\). Thus, the sequence is decreasing.

Since the sequence is both bounded below by \(\sqrt{2}\) and decreasing, it converges to a limit \(L\). Taking the limit as \(n \to \infty\) in the recurrence relation
\[
L = \frac{L^2 + 2}{2L},
\]
multiplying through by \(2L\) (where \(L > 0\)), we get
\[
2L^2 = L^2 + 2,
\]
which simplifies to
\[
L^2 = 2.
\]
Therefore,
\[
L = \sqrt{2}.
\]
The sequence \((a_n)\) converges to \(\sqrt{2}\).

\item 
For the differentiable function \(f(x)\) on \([n, n+1]\), the Mean Value Theorem guarantees the existence of some \(c_n \in (n, n+1)\) such that:
\[
f(n+1) - f(n) = f'(c_n).
\]
From the problem, we know that:
\[
f'(x) \to 0 \quad \text{as} \quad x \to \infty.
\]
Since \(c_n \in (n, n+1)\), as \(n \to \infty\), \(c_n \to \infty\). Hence, \(f'(c_n) \to 0\).

Therefore, as \(n \to \infty\),
\[
f(n+1) - f(n) = f'(c_n) \to 0.
\]
Using the result from the Mean Value Theorem and the given condition \(f'(x) \to 0\), we conclude:
\[
\lim_{n \to \infty} (f(n+1) - f(n)) = 0.
\]

\item Given the recurrence relation:
\[
a_1 = a, \quad a_{n+1} = \alpha a_n + \beta,
\]
we determine the conditions under which a limit exists and find the limit in two ways.

\textbf{Guess and Check Method:}

The recurrence relation is:
\[
a_{n+1} = \alpha a_n + \beta.
\]

Expanding the recurrence relation:
\[
a_2 = \alpha a_1 + \beta
\]
\[
a_3 = \alpha a_2 + \beta = \alpha(\alpha a_1 + \beta) + \beta = \alpha^2 a_1 + \alpha \beta + \beta,
\]
\[
a_4 = \alpha a_3 + \beta = \alpha(\alpha^2 a_1 + \alpha \beta + \beta) + \beta = \alpha^3 a_1 + \alpha^2 \beta + \alpha \beta + \beta.
\]
Following this pattern:
\[
a_n = \alpha^n a + \beta(1 + \alpha + \alpha^2 + \dots + \alpha^{n-1}).
\]
The second term is a geometric series. We know the sum formula for a finite geometric series as we were taught about it in class:
\[
1 + \alpha + \alpha^2 + \dots + \alpha^{n-1} = \frac{1 - \alpha^n}{1 - \alpha}, \quad \text{(for \(|\alpha| \neq 1\)}.
\]
Substituting this back, we get:
\[
a_n = \alpha^n a + \frac{\beta(1 - \alpha^n)}{1 - \alpha}.
\]

If \(|\alpha| < 1\):
\[
\lim_{n \to \infty} \alpha^n = 0.
\]
Thus:
\[
\lim_{n \to \infty} a_n = \frac{\beta}{1 - \alpha}.
\]

If \(|\alpha| \geq 1\), the term \(\alpha^n a\) does not converge to 0, and the sequence diverges.

Thus:
\[
\lim_{n \to \infty} a_n = \frac{\beta}{1 - \alpha}.
\]

\textbf{Using the Fixed Point Theorem:}

The recurrence relation:
\[
a_{n+1} = \alpha a_n + \beta
\]
defines a function \(f(x) = \alpha x + \beta\).

A fixed point \(L\) satisfies \(f(L) = L\):
\[
L = \alpha L + \beta.
\]
Solving for \(L\):
\[
L = \frac{\beta}{1 - \alpha}, \quad \text{(for \(|\alpha| \neq 1\)}.
\]

The Fixed Point Theorem states that the sequence \((a_n)\) will converge to the fixed point \(L\) if the derivative of \(f(x)\), which is \(|\alpha|\), satisfies \(|\alpha| < 1\). Thus, the condition for convergence is:
\[
|\alpha| < 1.
\]

Under the condition \(|\alpha| < 1\), the sequence converges to:
\[
\lim_{n \to \infty} a_n = \frac{\beta}{1 - \alpha}.
\]



\end{enumerate}

\end{document}
