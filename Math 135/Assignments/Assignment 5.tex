\documentclass[12pt]{article}
\usepackage{bigints}
\usepackage{graphicx}			% Use this package to include images
\usepackage{amsmath}	
\usepackage{amssymb}
\usepackage{amsfonts}
\usepackage{polynom}
\usepackage{listings}
% A library of many standard math expressions
\graphicspath{ {./Images/} }
\usepackage[margin=1in]{geometry}% Sets 1in margins. 
\newcommand{\qed}[0]{$\blacksquare$}
\usepackage{fancyhdr}			% Creates headers and footers
\usepackage{enumerate}          %These two package give custom labels to a list
\usepackage[shortlabels]{enumitem}


% Creates the header and footer. You can adjust the look and feel of these here.
\pagestyle{fancy}
\fancyhead[l]{Aditya Gupta}
\fancyhead[c]{Math 135 Homework \#5}
\fancyhead[r]{\today}
\fancyfoot[c]{\thepage}
\renewcommand{\headrulewidth}{0.2pt} %Creates a horizontal line underneath the header
\setlength{\headheight}{15pt} %Sets enough space for the header
\begin{document}
\begin{enumerate}
\item 
Given the differential equation:
\[
\frac{dz}{dx} + \frac{x + 1}{x}z = \frac{6}{x}.
\]

Find the integrating factor \( \mu(x) \):

\[
\mu(x) = e^{\int \frac{x + 1}{x} dx} = e^{\int 1 + \frac{1}{x} dx} = e^{x} x.
\]

Multiply through by \( \mu(x) \):

\[
xe^x \frac{dz}{dx} + (x+1)e^x z = 6e^x.
\]

The left-hand side is a product derivative:

\[
\frac{d}{dx} \left( zxe^x \right) = 6e^x.
\]

Integrate both sides:

\[
\int \frac{d}{dx} \left( zxe^x \right) dx = \int 6e^x dx.
\]

\[
zxe^x = 6e^x + C.
\]

Solve for \( z \):

\[
z = \frac{6}{x} + \frac{C}{xe^x}.
\]

Revert to \( y \):

\[
y^{-2} = \frac{6}{x} + \frac{C}{xe^x}.
\]

Take the reciprocal and solve for \( y \):

\[
y^2 = \frac{x}{6 + \frac{C}{e^x}}.
\]

Thus, the final solution is:

\[
y = \pm \sqrt{\frac{x}{6 + \frac{C}{e^x}}}.
\]


\item 
\begin{enumerate}
    \item \[
y'' + p(t) y' + q(t) y = g(t)
\]

Given that \( y_1 \) is a known solution of the corresponding homogeneous equation:

\[
y'' + p(t) y' + q(t) y = 0
\]

we assume the second solution is of the form:

\[
y_2 = v y_1
\]

where \( v \) is an unknown function. Differentiating,

\[
y_2' = v' y_1 + v y_1'
\]

\[
y_2'' = v'' y_1 + 2 v' y_1' + v y_1''
\]

Substituting into the homogeneous equation,

\[
(v'' y_1 + 2 v' y_1' + v y_1'') + p(t)(v' y_1 + v y_1') + q(t) (v y_1) = 0
\]

Since \( y_1 \) satisfies the homogeneous equation,

\[
y_1'' + p y_1' + q y_1 = 0,
\]

this simplifies to

\[
v'' y_1 + 2 v' y_1' = 0
\]

Dividing by \( y_1 \) (assuming \( y_1 \neq 0 \)),

\[
v'' + 2 \frac{y_1'}{y_1} v' = 0
\]

Introducing \( u = v' \), so that \( u' = v'' \), this reduces to the first-order equation:

\[
u' + 2 \frac{y_1'}{y_1} u = 0
\]

Separating variables,

\[
\frac{du}{u} = -2 \frac{y_1'}{y_1} dt
\]

Integrating,

\[
\ln |u| = -2 \ln |y_1| + C
\]

\[
u = C y_1^{-2}
\]

\[
v' = C y_1^{-2}
\]

Integrating,

\[
v = C \int y_1^{-2} dt + C_2
\]

Thus, the second solution is:

\[
y_2 = y_1 \int y_1^{-2} dt
\]

For the particular solution, we assume:

\[
y_p = v y_1
\]

where \( v \) satisfies:

\[
v'' y_1 + 2 v' y_1' = g(t)
\]

Using \( u = v' \), we obtain:

\[
u' y_1 + 2 u y_1' = g(t)
\]

Multiplying by the integrating factor:

\[
\mu = e^{\int 2 \frac{y_1'}{y_1} dt} = y_1^2
\]

\[
\frac{d}{dt} (y_1^2 u) = g(t) y_1^2
\]

Integrating,

\[
y_1^2 u = \int g(t) y_1^2 dt
\]

\[
u = y_1^{-2} \int g(t) y_1^2 dt
\]

Integrating again,

\[
v = \int y_1^{-2} \int g(t) y_1^2 dt \, dt
\]

Thus, the particular solution is:

\[
y_p = y_1 \int y_1^{-2} \int g(t) y_1^2 dt \, dt
\]

The general solution to the non-homogeneous differential equation is:

\[
y = c_1 y_1 + c_2 y_2 + y_p
\]

where:

- \( y_1 \) is the given solution of the homogeneous equation,

- \( y_2 = y_1 \int y_1^{-2} dt \),

- \( y_p = y_1 \int y_1^{-2} \int g(t) y_1^2 dt \, dt \).

\item 
\[
x^2 y'' + x y' - y = x^2 e^{-x}
\]

with a known solution to the homogeneous equation:

\[
x^2 y'' + x y' - y = 0, \quad y_1 = x.
\]

We follow the **reduction of order** method.

First, we find a second solution \( y_2 \) of the homogeneous equation.  
We assume:

\[
y_2 = v y_1 = v x.
\]

Computing derivatives,

\[
y_2' = v' x + v,
\]

\[
y_2'' = v'' x + 2 v'.
\]

Substituting into the homogeneous equation:

\[
x^2 (v'' x + 2 v') + x (v' x + v) - (v x) = 0.
\]

Expanding,

\[
x^3 v'' + 2x^2 v' + x^2 v' + x v - x v = 0.
\]

\[
x^3 v'' + 3x^2 v' = 0.
\]

Dividing by \( x^2 \) (for \( x \neq 0 \)),

\[
x v'' + 3 v' = 0.
\]

Introducing \( u = v' \), so that \( u' = v'' \), we obtain:

\[
x u' + 3 u = 0.
\]

Separating variables,

\[
\frac{du}{u} = -\frac{3}{x} dx.
\]

Integrating,

\[
\ln |u| = -3 \ln |x| + C.
\]

\[
u = C x^{-3}.
\]

\[
v' = C x^{-3}.
\]

Integrating,

\[
v = \int C x^{-3} dx = C \frac{x^{-2}}{-2} = -\frac{C}{2} x^{-2}.
\]

Thus,

\[
y_2 = x v = x \left(-\frac{C}{2} x^{-2}\right) = -\frac{C}{2} x^{-1}.
\]

So the second solution is:

\[
y_2 = x^{-1}.
\]

Thus, the general solution to the homogeneous equation is:

\[
y_h = c_1 x + c_2 x^{-1}.
\]

Now, we find the particular solution \( y_p \).  
We assume:

\[
y_p = v(x) x.
\]

Computing derivatives,

\[
y_p' = v' x + v,
\]

\[
y_p'' = v'' x + 2 v'.
\]

Substituting into the non-homogeneous equation:

\[
x^2 (v'' x + 2 v') + x (v' x + v) - (v x) = x^2 e^{-x}.
\]

Expanding,

\[
x^3 v'' + 2x^2 v' + x^2 v' + x v - x v = x^2 e^{-x}.
\]

\[
x^3 v'' + 3x^2 v' = x^2 e^{-x}.
\]

Dividing by \( x^2 \):

\[
x v'' + 3 v' = e^{-x}.
\]

Multiplying by the integrating factor:

\[
\mu = e^{\int \frac{3}{x} dx} = x^3.
\]

\[
\frac{d}{dx} (x^3 v') = x^3 e^{-x}.
\]

Integrating both sides,

\[
x^3 v' = \int x^3 e^{-x} dx.
\]

Using integration by parts:

Let \( u = x^3 \), so that \( du = 3x^2 dx \),  
and let \( dv = e^{-x} dx \), so that \( v = -e^{-x} \).  

Applying integration by parts:

\[
\int x^3 e^{-x} dx = -x^3 e^{-x} + \int 3x^2 e^{-x} dx.
\]

Applying integration by parts to \( \int 3x^2 e^{-x} dx \):

Let \( u = x^2 \), so that \( du = 2x dx \),  
and let \( dv = e^{-x} dx \), so that \( v = -e^{-x} \).  

\[
\int x^2 e^{-x} dx = -x^2 e^{-x} + 2 \int x e^{-x} dx.
\]

Applying integration by parts to \( \int x e^{-x} dx \):

Let \( u = x \), so that \( du = dx \),  
and let \( dv = e^{-x} dx \), so that \( v = -e^{-x} \).  

\[
\int x e^{-x} dx = -x e^{-x} + \int e^{-x} dx.
\]

\[
= -x e^{-x} - e^{-x}.
\]

Thus,

\[
\int x^2 e^{-x} dx = -x^2 e^{-x} - 2x e^{-x} - 2e^{-x}.
\]

\[
\int x^3 e^{-x} dx = -x^3 e^{-x} + 3(-x^2 e^{-x} - 2x e^{-x} - 2e^{-x}).
\]

\[
= -x^3 e^{-x} + 3x^2 e^{-x} + 6x e^{-x} + 6e^{-x}.
\]

\[
= e^{-x} (-x^3 + 3x^2 + 6x + 6).
\]

Thus,

\[
x^3 v' = e^{-x} (-x^3 + 3x^2 + 6x + 6).
\]

Dividing by \( x^3 \):

\[
v' = e^{-x} \left(-1 + \frac{3}{x} + \frac{6}{x^2} + \frac{6}{x^3} \right).
\]

Integrating,

\[
v = -\int e^{-x} dx + 3 \int \frac{e^{-x}}{x} dx + 6 \int \frac{e^{-x}}{x^2} dx + 6 \int \frac{e^{-x}}{x^3} dx.
\]

Thus,

\[
y_p = x v(x).
\]

The final general solution is:

\[
y = c_1 x + c_2 x^{-1} + x e^{-x} + x \left( -\int e^{-x} dx + 3 \int \frac{e^{-x}}{x} dx + 6 \int \frac{e^{-x}}{x^2} dx + 6 \int \frac{e^{-x}}{x^3} dx \right).
\]
\end{enumerate}
\item 
Let \( y_1 \) and \( y_2 \) be a fundamental set of solutions to the second-order linear differential equation 
\[
y'' + p(t)y' + q(t)y = 0,
\]
on the interval \((- \infty, \infty)\), where \( p(t) \) and \( q(t) \) are continuous functions. We aim to prove that there is exactly one zero of \( y_1 \) between two consecutive zeros of \( y_2 \).

Define the function \( f(t) = \frac{y_2(t)}{y_1(t)} \). Differentiating \( f(t) \) gives
\[
f'(t) = \frac{y_1(t)y_2'(t) - y_2(t)y_1'(t)}{y_1(t)^2}.
\]
The numerator of \( f'(t) \) is the Wronskian:
\[
W(t) = y_1(t)y_2'(t) - y_2(t)y_1'(t),
\]
and hence
\[
f'(t) = \frac{W(t)}{y_1(t)^2}.
\]

By Abel's identity, the Wronskian satisfies the equation
\[
W'(t) + p(t)W(t) = 0,
\]
which implies
\[
W(t) = Ce^{-\int p(t) \, dt},
\]
where \( C \) is a constant. Since \( y_1 \) and \( y_2 \) form a fundamental set of solutions, \( C \neq 0 \), and thus \( W(t) \neq 0 \) for all \( t \).

Suppose \( y_1 \) has more than one zero between two consecutive zeros of \( y_2 \). Let \( t_1 \) and \( t_2 \) be two such zeros of \( y_1 \) with \( t_1 < t_2 \). Since \( y_2(t) = 0 \) at the consecutive zeros, \( f(t) \) is zero at \( t_1 \) and \( t_2 \). By Rolle's Theorem, \( f'(t)=0 \) at some \( t \in (t_1, t_2) \).

However, \( f'(t) = \frac{W(t)}{y_1(t)^2} \), and \( W(t) \neq 0 \), so \( f'(t) \neq 0 \). This contradiction shows that \( y_1 \) cannot have more than one zero between two consecutive zeros of \( y_2 \).

Similarly, \( y_1 \) must have at least one zero between two consecutive zeros of \( y_2 \) because \( f(t) \) changes sign as \( y_2(t) \) crosses zero.

Thus, there is exactly one zero of \( y_1 \) between two consecutive zeros of \( y_2 \).

\item Let \( y_1 \) and \( y_2 \) be solutions to the second-order linear differential equation 
\[
y'' + py' + qy = 0,
\]
where \( p \) and \( q \) are continuous functions on the interval \( I = (a, b) \).

Assume that \( y_1 \) and \( y_2 \) either both vanish at the same point \( t_0 \), or both have maxima or minima at \( t_0 \).

The Wronskian of \( y_1 \) and \( y_2 \) is defined as
\[
W(t) = y_1(t)y_2'(t) - y_2(t)y_1'(t).
\]
By Abel's identity, the Wronskian satisfies
\[
W'(t) + p(t)W(t) = 0,
\]
which implies
\[
W(t) = Ce^{-\int p(t) \, dt},
\]
where \( C \) is a constant.

If \( y_1(t_0) = y_2(t_0) = 0 \), then at \( t_0 \),
\[
W(t_0) = y_1(t_0)y_2'(t_0) - y_2(t_0)y_1'(t_0) = 0.
\]
Since \( e^{-\int p(t) \, dt} \neq 0 \), the Wronskian \( W(t) \) is always zero on the interval \( I \). Thus, \( y_1 \) and \( y_2 \) are linearly dependent.

Similarly, if \( y_1 \) and \( y_2 \) both have extrema at \( t_0 \), then \( y_1'(t_0) = 0 \) and \( y_2'(t_0) = 0 \). At \( t_0 \),
\[
W(t_0) = y_1(t_0)y_2'(t_0) - y_2(t_0)y_1'(t_0) = 0.
\]
Again, \( W(t) = 0 \) for all \( t \), implying that \( y_1 \) and \( y_2 \) are linearly dependent.

Hence, in either case, one solution is a multiple of the other.

\end{enumerate}




\end{document}
