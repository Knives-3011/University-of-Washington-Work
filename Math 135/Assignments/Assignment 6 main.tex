\documentclass[12pt]{article}
\usepackage{bigints}
\usepackage{graphicx}			% Use this package to include images
\usepackage{amsmath}	
\usepackage{amssymb}
\usepackage{amsfonts}
\usepackage{polynom}
\usepackage{listings}
% A library of many standard math expressions
\graphicspath{ {./Images/} }
\usepackage[margin=1in]{geometry}% Sets 1in margins. 
\newcommand{\qed}[0]{$\blacksquare$}
\usepackage{fancyhdr}			% Creates headers and footers
\usepackage{enumerate}          %These two package give custom labels to a list
\usepackage[shortlabels]{enumitem}


% Creates the header and footer. You can adjust the look and feel of these here.
\pagestyle{fancy}
\fancyhead[l]{Aditya Gupta}
\fancyhead[c]{Math 135 Homework \#6}
\fancyhead[r]{\today}
\fancyfoot[c]{\thepage}
\renewcommand{\headrulewidth}{0.2pt} %Creates a horizontal line underneath the header
\setlength{\headheight}{15pt} %Sets enough space for the header
\begin{document}
\begin{enumerate}
\item 
\[
2y'' - 2y' + 3y = \sin t, \quad y(0) = -1, \quad y'(0) = 2.
\]

The corresponding homogeneous equation is:

\[
2y'' - 2y' + 3y = 0.
\]

Assuming a solution of the form \( y_h = e^{rt} \), the characteristic equation is:

\[
2r^2 - 2r + 3 = 0.
\]

Solving for \( r \) using the quadratic formula:

\[
r = \frac{-(-2) \pm \sqrt{(-2)^2 - 4(2)(3)}}{2(2)}
= \frac{2 \pm \sqrt{4 - 24}}{4}
= \frac{2 \pm \sqrt{-20}}{4}
= \frac{2 \pm 2i\sqrt{5}}{4}
= \frac{1 \pm i\sqrt{5}}{2}.
\]

Thus, the general solution of the homogeneous equation is:

\[
y_h = e^{t/2} \left( C_1 \cos \frac{\sqrt{5}}{2}t + C_2 \sin \frac{\sqrt{5}}{2}t \right).
\]

For a particular solution, assume:

\[
y_p = A \cos t + B \sin t.
\]

Computing derivatives:

\[
y_p' = -A \sin t + B \cos t, \quad y_p'' = -A \cos t - B \sin t.
\]

Substituting into the differential equation:

\[
2(-A \cos t - B \sin t) - 2(-A \sin t + B \cos t) + 3(A \cos t + B \sin t) = \sin t.
\]

Expanding:

\[
-2A \cos t - 2B \sin t + 2A \sin t - 2B \cos t + 3A \cos t + 3B \sin t = \sin t.
\]

Grouping terms:

\[
(-2A - 2B + 3A) \cos t + (-2B + 2A + 3B) \sin t = \sin t.
\]

Equating coefficients:

\[
A - 2B = 0 \Rightarrow A = 2B, \quad 2A + B = 1.
\]

Substituting \( A = 2B \) into the second equation:

\[
2(2B) + B = 1 \Rightarrow 5B = 1 \Rightarrow B = \frac{1}{5}.
\]

Thus,

\[
A = \frac{2}{5}.
\]

So the particular solution is:

\[
y_p = \frac{2}{5} \cos t + \frac{1}{5} \sin t.
\]

The general solution is:

\[
y(t) = e^{t/2} \left( C_1 \cos \frac{\sqrt{5}}{2} t + C_2 \sin \frac{\sqrt{5}}{2} t \right) + \frac{2}{5} \cos t + \frac{1}{5} \sin t.
\]

Applying initial conditions \( y(0) = -1 \) and \( y'(0) = 2 \), we get:

\[
C_1 + \frac{2}{5} = -1 \Rightarrow C_1 = -\frac{7}{5}.
\]

Computing \( y'(t) \):

\[
y_h' = e^{t/2} \left( \frac{1}{2} C_1 \cos \frac{\sqrt{5}}{2} t + \frac{1}{2} C_2 \sin \frac{\sqrt{5}}{2} t \right) + e^{t/2} \left( -C_1 \frac{\sqrt{5}}{2} \sin \frac{\sqrt{5}}{2} t + C_2 \frac{\sqrt{5}}{2} \cos \frac{\sqrt{5}}{2} t \right).
\]

At \( t = 0 \):

\[
y_h'(0) = \frac{1}{2} C_1 + C_2 \frac{\sqrt{5}}{2}.
\]

Derivative of \( y_p \):

\[
y_p' = -\frac{2}{5} \sin t + \frac{1}{5} \cos t.
\]

At \( t = 0 \):

\[
y_p'(0) = \frac{1}{5}.
\]

Thus,

\[
\frac{1}{2} C_1 + C_2 \frac{\sqrt{5}}{2} + \frac{1}{5} = 2.
\]

Substituting \( C_1 = -\frac{7}{5} \):

\[
\frac{1}{2} \left(-\frac{7}{5}\right) + C_2 \frac{\sqrt{5}}{2} + \frac{1}{5} = 2.
\]

\[
-\frac{7}{10} + C_2 \frac{\sqrt{5}}{2} + \frac{1}{5} = 2.
\]

\[
C_2 \frac{\sqrt{5}}{2} = 2 + \frac{7}{10} - \frac{1}{5}.
\]

\[
C_2 \frac{\sqrt{5}}{2} = \frac{20}{10} + \frac{7}{10} - \frac{2}{10} = \frac{25}{10} = \frac{5}{2}.
\]

\[
C_2 = \frac{5}{\sqrt{5}} = \sqrt{5}.
\]

Thus, the final solution is:

\[
y(t) = e^{t/2} \left( -\frac{7}{5} \cos \frac{\sqrt{5}}{2} t + \sqrt{5} \sin \frac{\sqrt{5}}{2} t \right) + \frac{2}{5} \cos t + \frac{1}{5} \sin t.
\]

\item 
The given differential equation is:

\[
y''' - 2y'' + y' = 2e^x + 2x
\]

The characteristic equation of the homogeneous part is:

\[
r^3 - 2r^2 + r = 0
\]

Factorizing,

\[
r (r - 1)^2 = 0
\]

which gives the roots:

\[
r_1 = 0, \quad r_2 = 1, \quad r_3 = 1
\]

The general solution of the homogeneous equation is:

\[
y_h = C_1 + (C_2 + C_3 x) e^x
\]

To find a particular solution, consider the right-hand side \( 2e^x + 2x \). Since \( e^x \) appears in the homogeneous solution with multiplicity 2, assume:

\[
y_{p1} = A x^2 e^x
\]

Taking derivatives:

\[
y_{p1}' = (A x^2 + 2A x) e^x
\]

\[
y_{p1}'' = (A x^2 + 4A x + 2A) e^x
\]

\[
y_{p1}''' = (A x^2 + 6A x + 6A) e^x
\]

Substituting into the differential equation:

\[
(A x^2 + 6A x + 6A) e^x - 2(A x^2 + 4A x + 2A) e^x + (A x^2 + 2A x) e^x = 2 e^x
\]

\[
(A - 2A + A) x^2 e^x + (6A - 8A + 2A) x e^x + (6A - 4A) e^x = 2 e^x
\]

\[
0 x^2 e^x + 0 x e^x + 2A e^x = 2 e^x
\]

\[
2A = 2 \Rightarrow A = 1
\]

Thus,

\[
y_{p1} = x^2 e^x
\]

For the term \( 2x \), assume:

\[
y_{p2} = B x + C
\]

Differentiating:

\[
y_{p2}' = B, \quad y_{p2}'' = 0, \quad y_{p2}''' = 0
\]

Substituting into the equation:

\[
2Ax + B - 4A = 2x
\]

Equating coefficients:

\[
2A = 2 \Rightarrow A = 1, \quad B - 4A = 2 \Rightarrow B - 4 = 2 \Rightarrow B = 4
\]

\[
y_{p2} = 4x
\]

Thus, the general solution to the differential equation is:

\[
y = (x^2 + C_1 x + C) e^x + x^2 + 4x + C_2
\]

Given the initial conditions:

\[
y(0) = 0, \quad y'(0) = 0, \quad y''(0) = 0
\]

Substituting \( x = 0 \) into the general solution:

\[
y(0) = (0^2 + C_1 \cdot 0 + C)e^0 + 0^2 + 4(0) + C_2 = 0
\]

\[
C + C_2 = 0
\]

First derivative:

\[
y' = (x^2 + C_1 x + C) e^x + (2x + C_1) e^x + x^2 + 4
\]

Substituting \( x = 0 \):

\[
C + C_1 + 4 = 0
\]

Second derivative:

\[
y'' = (x^2 + C_1 x + C) e^x + 2(2x + C_1) e^x + 2
\]

Substituting \( x = 0 \):

\[
2C_1 + C + 4 = 0
\]

Solving for \( C, C_1, C_2 \):

\[
C = -4, \quad C_1 = 0, \quad C_2 = 4
\]

Final solution:

\[
y = (x^2 - 4)e^x + x^2 + 4x + 4
\]

with initial conditions:

\[
y(0) = 0, \quad y'(0) = 0, \quad y''(0) = 0.
\]

\item 
\[
2x^2 y'' + 3x y' - y = x^{-1}
\]

The corresponding homogeneous equation is:

\[
2x^2 y'' + 3x y' - y = 0
\]

The given linearly independent solutions to the homogeneous equation are:

\[
y_1 = x^{1/2}, \quad y_2 = x^{-1}
\]

Thus, the general solution to the homogeneous equation is:

\[
y_h = C_1 x^{1/2} + C_2 x^{-1}
\]

Since the right-hand side of the equation is \( x^{-1} \), which matches the homogeneous solution \( y_2 = x^{-1} \), we modify our assumption by introducing a logarithmic factor:

\[
y_p = v(x) x^{-1}
\]

where \( v(x) \) is an unknown function to be determined.

The first derivative of \( y_p \) is:

\[
y_p' = v' x^{-1} - v x^{-2}
\]

and the second derivative is:

\[
y_p'' = v'' x^{-1} - 2 v' x^{-2} + 2 v x^{-3}
\]

Substituting these into the original differential equation:

\[
2x^2 (v'' x^{-1} - 2 v' x^{-2} + 2 v x^{-3}) + 3x (v' x^{-1} - v x^{-2}) - v x^{-1} = x^{-1}
\]

Expanding:

\[
2x v'' - 4x v' + 4v + 3v' - 3v - v = 1
\]

\[
2x v'' - x v' = 1
\]

Rewriting:

\[
v'' - \frac{1}{2x} v' = \frac{1}{2x}
\]

Let \( w = v' \), then:

\[
w' - \frac{1}{2x} w = \frac{1}{2x}
\]

The integrating factor is:

\[
IF = e^{-\int \frac{1}{2x} dx} = e^{-\frac{1}{2} \ln x} = x^{-1/2}
\]

Multiplying through by \( x^{-1/2} \):

\[
(w x^{-1/2})' = \frac{1}{2x} x^{-1/2}
\]

Integrating:

\[
w x^{-1/2} = \int \frac{1}{2} x^{-3/2} dx
\]

\[
= \int \frac{1}{2} x^{-3/2} dx = -x^{-1/2}
\]

\[
w = -\frac{1}{3} \ln x
\]

Since \( w = v' \), integrating again:

\[
v = -\frac{1}{3} x \ln x
\]

Thus, the particular solution is:

\[
y_p = v(x) x^{-1} = -\frac{1}{3} x^{-1} \ln x
\]

The final general solution is:

\[
y = C_1 x^{1/2} + C_2 x^{-1} - \frac{1}{3} x^{-1} \ln x
\]

\[
y = C_1 x^{1/2} + C_2 x^{-1} - \frac{1}{3} x^{-1} \ln x
\]

\end{enumerate}
\end{document}
