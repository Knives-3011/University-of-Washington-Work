\documentclass[12pt]{article}
\usepackage{bigints}
\usepackage{graphicx}			% Use this package to include images
\usepackage{amsmath}	
\usepackage{amssymb}
\usepackage{amsfonts}
\usepackage{polynom}
\usepackage{listings}
% A library of many standard math expressions
\graphicspath{ {./Images/} }
\usepackage[margin=1in]{geometry}% Sets 1in margins. 
\newcommand{\qed}[0]{$\blacksquare$}
\usepackage{fancyhdr}			% Creates headers and footers
\usepackage{enumerate}          %These two package give custom labels to a list
\usepackage[shortlabels]{enumitem}
% Creates the header and footer. You can adjust the look and feel of these here.
\pagestyle{fancy}
\fancyhead[l]{Aditya Gupta}
\fancyhead[c]{Math 135 Homework \#7}
\fancyhead[r]{\today}
\fancyfoot[c]{\thepage}
\renewcommand{\headrulewidth}{0.2pt} %Creates a horizontal line underneath the header
\setlength{\headheight}{15pt} %Sets enough space for the header
\begin{document}
\begin{enumerate}
\item 
A function \( f(t) \) is said to be of exponential order if there exist constants \( M > 0 \), \( c > 0 \), and \( T > 0 \) such that:

\[
|f(t)| \leq M e^{ct}, \quad \forall t > T.
\]

Now, suppose \( f(t) = e^{t^2} \) were of exponential order. Then, there would exist constants \( M, c, T \) such that:

\[
e^{t^2} \leq M e^{ct}, \quad \forall t > T.
\]

Taking the natural logarithm on both sides:

\[
t^2 \leq \ln M + ct.
\]

Rearranging:

\[
t^2 - ct \leq \ln M.
\]

For sufficiently large \( t \), specifically when \( t > c \), the left-hand side grows without bound because the quadratic term \( t^2 \) dominates the linear term \( ct \). This results in a contradiction, as no finite \( M \) can satisfy this inequality. 

Thus, no such constants \( M \) and \( c \) exist, proving that \( f(t) = e^{t^2} \) is not of exponential order.

\item 
\begin{enumerate}[label=(\alph*)]
\item We use the fact that
\[
\int e^{at}\cos(bt)\,dt=\Re\left\{\int e^{(a+ib)t}\,dt\right\}=\Re\left\{\frac{e^{(a+ib)t}}{a+ib}\right\},
\]
and by multiplying numerator and denominator by \(a-ib\) we obtain
\[
\Re\left\{\frac{e^{(a+ib)t}(a-ib)}{a^2+b^2}\right\}=\frac{e^{at}}{a^2+b^2}\Bigl[a\cos(bt)+b\sin(bt)\Bigr]+C.
\]
Thus,
\[
\int e^{at}\cos(bt)\,dt=\frac{e^{at}}{a^2+b^2}\Bigl[a\cos(bt)+b\sin(bt)\Bigr]+C.
\]

\item Starting from the definition
\[
L\{\cos(bt)\}=\int_0^\infty e^{-st}\cos(bt)\,dt,
\]
write
\[
\cos(bt)=\frac{e^{ibt}+e^{-ibt}}{2},
\]
so that
\[
L\{\cos(bt)\}=\frac{1}{2}\left(\int_0^\infty e^{-(s-ib)t}\,dt+\int_0^\infty e^{-(s+ib)t}\,dt\right)
=\frac{1}{2}\left(\frac{1}{s-ib}+\frac{1}{s+ib}\right).
\]
Combining the fractions yields
\[
\frac{1}{s-ib}+\frac{1}{s+ib}=\frac{2s}{s^2+b^2},
\]
and hence,
\[
L\{\cos(bt)\}=\frac{s}{s^2+b^2}.
\]

\item Using the definition
\[
L\{\sin(bt)\}=\int_0^\infty e^{-st}\sin(bt)\,dt,
\]
express
\[
\sin(bt)=\frac{e^{ibt}-e^{-ibt}}{2i},
\]
so that
\[
L\{\sin(bt)\}=\frac{1}{2i}\left(\int_0^\infty e^{-(s-ib)t}\,dt-\int_0^\infty e^{-(s+ib)t}\,dt\right)
=\frac{1}{2i}\left(\frac{1}{s-ib}-\frac{1}{s+ib}\right).
\]
Since
\[
\frac{1}{s-ib}-\frac{1}{s+ib}=\frac{2ib}{s^2+b^2},
\]
it follows that
\[
L\{\sin(bt)\}=\frac{b}{s^2+b^2}.
\]

\item To find \(L\{t\sin(bt)\}\) we use the differentiation property of the Laplace transform:
\[
L\{t\sin(bt)\}=-\frac{d}{ds}\Bigl(L\{\sin(bt)\}\Bigr)
=-\frac{d}{ds}\left(\frac{b}{s^2+b^2}\right).
\]
Differentiating yields
\[
\frac{d}{ds}\left(\frac{b}{s^2+b^2}\right)=-\frac{2bs}{(s^2+b^2)^2},
\]
so that
\[
L\{t\sin(bt)\}=\frac{2bs}{(s^2+b^2)^2}.
\]

\item By the first shifting theorem, for any function \(f(t)\) with Laplace transform \(F(s)\) we have
\[
L\{e^{at}f(t)\}=F(s-a).
\]
Since
\[
L\{\sin(bt)\}=\frac{b}{s^2+b^2},
\]
it follows that
\[
L\{e^{at}\sin(bt)\}=\frac{b}{(s-a)^2+b^2}.
\]

\item Using the differentiation property on the transform from part (e),
\[
L\{t\,e^{at}\sin(bt)\}=-\frac{d}{ds}\left(\frac{b}{(s-a)^2+b^2}\right).
\]
Differentiation gives
\[
-\frac{d}{ds}\left(\frac{b}{(s-a)^2+b^2}\right)
=\frac{2b(s-a)}{\bigl[(s-a)^2+b^2\bigr]^2},
\]
so that
\[
L\{t\,e^{at}\sin(bt)\}=\frac{2b(s-a)}{\bigl[(s-a)^2+b^2\bigr]^2}.
\]

\item Write
\[
\frac{s^2}{(s^2+a^2)^2}=\frac{s^2+a^2-a^2}{(s^2+a^2)^2}
=\frac{1}{s^2+a^2}-\frac{a^2}{(s^2+a^2)^2}.
\]
Recall that
\[
L^{-1}\Bigl\{\frac{1}{s^2+a^2}\Bigr\}=\frac{1}{a}\sin(at)
\]
and that a standard result is
\[
L^{-1}\Bigl\{\frac{1}{(s^2+a^2)^2}\Bigr\}=\frac{1}{2a^3}\Bigl[\sin(at)-at\cos(at)\Bigr].
\]
Thus,
\[
L^{-1}\Bigl\{\frac{s^2}{(s^2+a^2)^2}\Bigr\}
=\frac{1}{a}\sin(at)-\frac{a^2}{2a^3}\Bigl[\sin(at)-at\cos(at)\Bigr]
=\frac{1}{2a}\Bigl[\sin(at)+at\cos(at)\Bigr],
\]
and similarly,
\[
L^{-1}\Bigl\{\frac{a^2}{(s^2+a^2)^2}\Bigr\}
=\frac{1}{2a}\Bigl[\sin(at)-at\cos(at)\Bigr].
\]
\end{enumerate}

\item 
\[
y'' + y = f(t), \quad y(0) = 0, \quad y'(0) = 0,
\]

where

\[
f(t) =
\begin{cases} 
4, & 0 \leq t \leq 2, \\
t + 2, & t > 2.
\end{cases}
\]

Taking the Laplace transform of both sides:

\[
L\{y''\} + L\{y\} = L\{f(t)\}.
\]

Using the properties:

\[
L\{y''\} = s^2 Y(s) - sy(0) - y'(0) = s^2 Y(s),
\]

\[
L\{y\} = Y(s),
\]

we obtain:

\[
(s^2 + 1) Y(s) = L\{f(t)\}.
\]

Thus,

\[
Y(s) = \frac{L\{f(t)\}}{s^2 + 1}.
\]


We compute the Laplace transform as:

\[
L\{f(t)\} = \int_0^\infty e^{-st} f(t) dt.
\]

Splitting the integral at \( t = 2 \):

\[
L\{f(t)\} = \int_0^2 4 e^{-st} dt + \int_2^\infty (t+2) e^{-st} dt.
\]


\[
\int_0^2 4 e^{-st} dt = 4 \left[ \frac{e^{-st}}{-s} \right]_0^2 = 4 \left( \frac{1 - e^{-2s}}{s} \right).
\]

Using integration by parts:

Let

- \( u = t+2 \) so that \( du = dt \),
- \( dv = e^{-st} dt \) so that \( v = \frac{e^{-st}}{-s} \).

Applying integration by parts:

\[
\int_2^\infty (t+2) e^{-st} dt = \left[ (t+2) \frac{e^{-st}}{-s} \right]_2^\infty - \int_2^\infty \frac{e^{-st}}{-s} dt.
\]

Evaluating the boundary terms,

\[
\lim_{t \to \infty} (t+2) \frac{e^{-st}}{-s} = 0.
\]

At \( t = 2 \):

\[
(2+2) \frac{e^{-2s}}{-s} = -\frac{4e^{-2s}}{s}.
\]

For the remaining integral:

\[
\int_2^\infty \frac{e^{-st}}{-s} dt = \frac{e^{-2s}}{s^2}.
\]

Thus,

\[
\int_2^\infty (t+2) e^{-st} dt = -\frac{4e^{-2s}}{s} + \frac{e^{-2s}}{s^2} = e^{-2s} \left( \frac{1}{s^2} - \frac{4}{s} \right).
\]

Solving for \( Y(s) \)

\[
L\{f(t)\} = 4 \frac{1 - e^{-2s}}{s} + e^{-2s} \left( \frac{1}{s^2} - \frac{4}{s} \right).
\]

\[
Y(s) = \frac{4(1 - e^{-2s})}{s(s^2+1)} + e^{-2s} \frac{\frac{1}{s^2} - \frac{4}{s}}{s^2+1}.
\]

Using inverse Laplace transform techniques, we find:

\[
L^{-1} \left[ \frac{4}{s(s^2+1)} \right] = 4 - 4\cos t.
\]

Using the time shift property:

\[
L^{-1} \left[ \frac{-4e^{-2s}}{s(s^2+1)} \right] = \alpha(t-2)(4 - 4\cos(t-2)).
\]

For the second term:

\[
L^{-1} \left[ e^{-2s} \frac{1/s^2 - 4/s}{s^2+1} \right] = \alpha(t-2) ((t-2) - 4).
\]

Thus:

\[
y(t) =
\begin{cases} 
4 - 4\cos t, & 0 \leq t \leq 2, \\
\alpha(t-2) \left[ (t-2) - 4 + 4 - 4\cos(t-2) \right], & t > 2.
\end{cases}
\]

Simplifying,

\[
y(t) =
\begin{cases} 
4 - 4\cos t, & 0 \leq t \leq 2, \\
t - 2 - 4\cos(t-2), & t > 2.
\end{cases}
\]

\item 
\[
h(t) =
\begin{cases} 
t, & 0 \leq t < 1, \\
h(t - 1), & t \geq 1.
\end{cases}
\]

Since \( h(t) \) is periodic with period \( T = 1 \), its Laplace transform is given by:

\[
L\{ h(t) \} = \frac{\int_0^T e^{-st} h(t) dt}{1 - e^{-sT}}.
\]

Computing the integral:

\[
I = \int_0^1 t e^{-st} dt.
\]

Using integration by parts, let:

\[
u = t, \quad dv = e^{-st} dt.
\]

Then:

\[
du = dt, \quad v = \frac{-e^{-st}}{s}.
\]

Applying integration by parts:

\[
I = \left[ t \cdot \frac{-e^{-st}}{s} \right]_0^1 - \int_0^1 \frac{-e^{-st}}{s} dt.
\]

Evaluating the first term:

\[
I = -\frac{e^{-s}}{s} + \frac{1}{s} \int_0^1 e^{-st} dt.
\]

Since:

\[
\int_0^1 e^{-st} dt = \frac{1 - e^{-s}}{s},
\]

substituting,

\[
I = -\frac{e^{-s}}{s} + \frac{1 - e^{-s}}{s^2}.
\]

Simplifying:

\[
I = \frac{1 - e^{-s}(1 + s)}{s^2}.
\]

Now, using the periodic Laplace transform formula:

\[
L\{ h(t) \} = \frac{I}{1 - e^{-s}},
\]

substituting \( I \):

\[
L\{ h(t) \} = \frac{\frac{1 - e^{-s} (1 + s)}{s^2}}{1 - e^{-s}}.
\]

Factoring \( (1 - e^{-s}) \) in the numerator:

\[
L\{ h(t) \} = \frac{(1 - e^{-s}) - s e^{-s}}{s^2 (1 - e^{-s})}.
\]

Canceling \( (1 - e^{-s}) \):

\[
L\{ h(t) \} = \frac{1 - s e^{-s}}{s^2 (1 - e^{-s})}.
\]

\end{enumerate}
\end{document}
